%%%%%%%%%%%%%%%%%%%%%%%%%%%%%%%%%%%%%%%%%%%%%%%%%%%%%%%%%%%%%%%%%%%%%%%%%%%%%%%
% PREAMBOLO COMUNE PER APPUNTI (Stile Scuro)
%
% Questo file contiene tutte le impostazioni e i pacchetti comuni.
% NON contiene \begin{document} o \end{document}.
%
% Istruzioni per la compilazione del file principale:
% pdflatex -shell-escape nomefile_principale.tex
%%%%%%%%%%%%%%%%%%%%%%%%%%%%%%%%%%%%%%%%%%%%%%%%%%%%%%%%%%%%%%%%%%%%%%%%%%%%%%%

\documentclass{article}

% --- Encoding e lingua ---
\usepackage[utf8]{inputenc}
\usepackage[italian]{babel}

% --- Margini e layout ---
\usepackage{geometry}
\geometry{a4paper, margin=1in}

% --- Font sans-serif (come Helvetica) ---
\usepackage[scaled]{helvet}
\renewcommand{\familydefault}{\sfdefault}
\usepackage[T1]{fontenc}

% --- Matematica ---
\usepackage{amsmath}
\usepackage{amssymb}

% --- Liste personalizzate ---
\usepackage{enumitem}
% \setlist{nosep}

% --- Immagini e Grafica ---
\usepackage{float}
% \usepackage{graphicx}
\usepackage{tikz}
\usetikzlibrary{shapes.geometric, positioning, arrows.meta, calc, fit, backgrounds, patterns, decorations.pathreplacing}

% --- Tabelle Avanzate ---
\usepackage{array}
\usepackage{booktabs}
\usepackage{longtable}

\usepackage{siunitx} % Per unità di misura come GHz, dBm, ecc.

% --- Hyperlink e Metadati PDF ---
\usepackage{hyperref}

\usepackage{graphicx}

\hypersetup{
    colorlinks=true,
    linkcolor=white,
    filecolor=magenta,
    urlcolor=cyan,
    citecolor=green,
    % pdftitle, pdfauthor, ecc. verranno impostati nel file principale
    pdfpagemode=FullScreen,
    bookmarksopen=true,
    bookmarksnumbered=true
}

% --- Licenza del documento ---
\usepackage[
  type={CC},
  modifier={by-sa},
  version={4.0},
]{doclicense}

% --- Colori e Sfondo Nero ---
\usepackage{xcolor}
\pagecolor{black}
\color{white}

% --- Evidenziazione del Codice ---
\usepackage{minted}
\setminted{
    frame=lines,
    framesep=2mm,
    fontsize=\small,
    breaklines=true,
    style=monokai,
    bgcolor=black!80
}
\usemintedstyle{monokai}

% --- Comandi personalizzati per algebra relazionale ---
\newcommand{\Rel}[1]{\textit{#1}} % Per i nomi delle relazioni
\newcommand{\Attr}[1]{\textsf{#1}} % Per i nomi degli attributi

\newcommand{\myunion}{\cup}
\newcommand{\myintersection}{\cap}
\newcommand{\mydifference}{-}
\newcommand{\myrename}[2]{\rho_{#1}(#2)}
\newcommand{\myselectop}[2]{\sigma_{#1}(#2)}
\newcommand{\myproject}[2]{\pi_{#1}(#2)}
\newcommand{\mycartesian}{\times}
\newcommand{\mynaturaljoin}{\bowtie} % Usare \Join da amssymb se disponibile e preferito
\newcommand{\mythetajoin}[3]{#1 \bowtie_{#2} #3} % R1 \bowtie_cond R2

% --- Comandi personalizzati per logica ---
\newcommand{\mylandop}{\wedge}
\newcommand{\myvel}{\vee}
\newcommand{\mynegop}{\neg}
\newcommand{\myforallop}{\forall}
\newcommand{\myexistsop}{\exists}

% --- Join esterni (outer join) ---
% Definizione standard per i join esterni
\def\ojoin{\setbox0=\hbox{$\mynaturaljoin$}%
	\rule[-.02ex]{.25em}{.4pt}\llap{\rule[\ht0]{.25em}{.4pt}}}
\newcommand{\myleftouterjoin}{\mathbin{\ojoin\mkern-5.8mu\mynaturaljoin}}
\newcommand{\myrightouterjoin}{\mathbin{\mynaturaljoin\mkern-5.8mu\ojoin}}
\newcommand{\myfullouterjoin}{\mathbin{\ojoin\mkern-5.8mu\mynaturaljoin\mkern-5.8mu\ojoin}}



% --- Titolo ---
\title{Reti Wireless: Routing e Trasporto}
\author{Basato sulle slide del Prof. Luciano Bononi}
\date{\today}

\begin{document}

\maketitle
\tableofcontents
\newpage

\section{Mobile Ad Hoc Networks (MANET)}

Le MANET (Mobile Ad Hoc Networks) sono reti formate da \textbf{dispositivi (host) wireless che possono essere mobili}. La caratteristica principale è che \textbf{non necessitano (obbligatoriamente) di un'infrastruttura preesistente} come access point o router centrali.

\subsection{Caratteristiche Fondamentali}
\begin{itemize}
    \item \textbf{Formate da host wireless mobili:} I nodi possono muoversi liberamente.
    \item \textbf{Senza infrastruttura preesistente (o opzionale):} I nodi comunicano direttamente tra loro (peer-to-peer) o attraverso altri nodi che fungono da router.
        \begin{itemize}
            \item \textit{Esempio pratico:} Un gruppo di persone con smartphone in un'area senza copertura WiFi/cellulare che vogliono scambiarsi messaggi o file. Ogni telefono può inoltrare i dati per gli altri.
        \end{itemize}
    \item \textbf{Multi-hop route:} Per raggiungere una destinazione, un pacchetto potrebbe dover attraversare più nodi intermedi.
    \item \textbf{La mobilità causa cambiamenti di route:} Se un nodo intermedio si sposta, la route precedentemente valida potrebbe non esserlo più, richiedendo la scoperta di una nuova route.
\end{itemize}

\begin{figure}[H]
    \centering
    \begin{tikzpicture}[node distance=1.5cm and 2cm]
        \node[node_style] (S) {S};
        \node[node_style, right=of S] (I1) {I1};
        \node[node_style, right=of I1] (D) {D};
        \node[node_style, below=of I1] (I2) {I2};

        \draw[highlight_link_style] (S) -- (I1);
        \draw[highlight_link_style] (I1) -- (D);
        \draw[link_style, dashed] (S) -- (I2);
        \draw[link_style, dashed] (I2) -- (D);

        \node[info_box, below=of I2, yshift=-0.5cm] {Pacchetto da S a D. Route S-I1-D. Se I1 si sposta, potrebbe essere necessaria la route S-I2-D.};
    \end{tikzpicture}
    \caption{Esempio di route multi-hop e cambiamento di route in una MANET.}
    \label{fig:manet_multihop}
\end{figure}

\section{Routing Unicast in MANET}
Il routing in MANET è diverso da quello nelle reti cablate tradizionali a causa di:
\begin{enumerate}
    \item \textbf{Mobilità degli Host:} I fallimenti e le riparazioni dei link dovuti alla mobilità hanno caratteristiche diverse.
    \item \textbf{Alta Frequenza di Fallimento/Riparazione dei Link:} La topologia della rete cambia rapidamente.
    \item \textbf{Nuovi Criteri di Performance:}
    \begin{itemize}
        \item Stabilità della route vs. mobilità.
        \item Consumo energetico (nodi spesso a batteria).
    \end{itemize}
\end{enumerate}

\subsection{Tipologie di Protocolli di Routing Unicast}
Molti protocolli sono stati proposti, ma nessuno è ottimo in tutti gli scenari. Si tenta di sviluppare protocolli \textit{adattivi}.
Le principali categorie sono:
\begin{itemize}
    \item \textbf{Protocolli Proattivi (o Table-Driven):}
    \begin{itemize}
        \item Determinano e mantengono le route verso tutte le destinazioni \textbf{indipendentemente dal traffico}.
        \item Ogni nodo ha una tabella di routing aggiornata.
        \item Esempi: Protocolli basati su \textit{link-state} o \textit{distance-vector}, adattati per MANET.
        \item \textit{Esempio:} Ogni nodo invia periodicamente informazioni sulla sua connettività ai vicini, permettendo a tutti di costruire una mappa della rete.
    \end{itemize}
    \item \textbf{Protocolli Reattivi (o On-Demand):}
    \begin{itemize}
        \item Trovano e mantengono route \textbf{solo quando necessario}.
        \item \textit{Esempio:} Se il nodo A vuole inviare al nodo Z, e non sa come raggiungerlo, A avvia un processo di scoperta della route.
    \end{itemize}
    \item \textbf{Protocolli Ibridi:}
    \begin{itemize}
        \item Combinano aspetti dei proattivi e dei reattivi.
    \end{itemize}
\end{itemize}

\subsection{Trade-Off tra Proattivi e Reattivi}
\begin{table}[H]
    \centering
    \begin{tabular}{|l|p{5cm}|p{5cm}|}
        \hline
        \rowcolor{bg_custom}
        \textbf{Caratteristica} & \textbf{Protocolli Proattivi} & \textbf{Protocolli Reattivi} \\
        \hline
        Latenza Scoperta Route & Bassa (route già note) & Alta (route da scoprire su richiesta) \\
        \hline
        Overhead (Controllo) & Alto (aggiornamenti continui) & Basso (solo per route attive) \\
        \hline
        Ideale per... & Reti con bassa mobilità e traffico frequente. & Reti con alta mobilità e traffico sporadico. \\
        \hline
    \end{tabular}
    \caption{Confronto tra protocolli proattivi e reattivi.}
    \label{tab:pro_vs_rea}
\end{table}
La scelta dell'approccio migliore dipende dai pattern di traffico e mobilità della rete.

\section{Panoramica dei Protocolli di Routing Unicast}

\subsection{Flooding per la Consegna dei Dati}
Meccanismo semplice ma spesso inefficiente.
\begin{itemize}
    \item \textbf{Funzionamento:}
    \begin{enumerate}
        \item Il mittente S trasmette (broadcast) il pacchetto P a tutti i suoi vicini.
        \item Ogni nodo che riceve P (e non è la destinazione e non l'ha già inoltrato) lo inoltra a sua volta a tutti i suoi vicini.
        \item Si usano \textbf{numeri di sequenza} per evitare loop o inoltri multipli.
        \item Il pacchetto P raggiunge la destinazione D se D è raggiungibile da S.
        \item Il nodo D, una volta ricevuto P, non lo inoltra ulteriormente.
    \end{enumerate}
    \item \textbf{Problemi Illustrati:}
    \begin{itemize}
        \item \textit{Potenziale collisione:} Se due nodi trasmettono contemporaneamente allo stesso vicino.
        \item \textit{Problema del nodo nascosto:} Due nodi potrebbero non "vedersi" ma entrambi trasmettere a un terzo nodo, causando collisione su quest'ultimo.
        \item \textit{Broadcast storm:} Troppi nodi inoltrano il pacchetto.
    \end{itemize}
    \item \textbf{Vantaggi:} Semplicità, potenziale alta affidabilità su percorsi multipli (se le collisioni non dominano), efficiente con topologia molto dinamica e basso rate.
    \item \textbf{Svantaggi:} Overhead potenzialmente molto alto, bassa affidabilità di consegna a causa della natura unreliable del broadcast e delle collisioni.
\end{itemize}

\subsection{Flooding di Pacchetti di Controllo}
Molti protocolli usano il flooding in modo più intelligente:
\begin{itemize}
    \item Flooding (potenzialmente limitato) di \textbf{pacchetti di controllo} (piccoli) invece dei pacchetti dati.
    \item I pacchetti di controllo scoprono le route.
    \item L'overhead è \textbf{ammortizzato} sui pacchetti dati che usano la route.
\end{itemize}

\subsection{Dynamic Source Routing (DSR)}
Protocollo \textbf{reattivo} che usa il \textbf{source routing}.
\begin{itemize}
    \item \textbf{Route Discovery:}
    \begin{itemize}
        \item S invia in flooding un \texttt{RREQ} (Route Request).
        \item Ogni nodo intermedio aggiunge il proprio ID al pacchetto \texttt{RREQ}.
        \item \textit{Esempio:} \texttt{RREQ} da S, inoltrato da E arriva a F come \texttt{[S,E]}. F lo inoltra a J, che lo riceve come \texttt{[S,E,F]}.
    \end{itemize}
    \item \textbf{Route Reply:}
    \begin{itemize}
        \item D invia un \texttt{RREP} (Route Reply) a S.
        \item L'\texttt{RREP} contiene la route completa (es. \texttt{[S,E,F,J,D]}) e torna indietro invertendo la route dell'\texttt{RREQ}. Presume link \textbf{bidirezionali}.
    \end{itemize}
    \item \textbf{Invio Dati:}
    \begin{itemize}
        \item S include l'\textbf{intera route nell'header} del pacchetto dati.
        \item Svantaggio: header cresce con la lunghezza della route.
    \end{itemize}
    \item \textbf{Ottimizzazione: Route Caching:}
    \begin{itemize}
        \item Ogni nodo mantiene una cache delle route apprese.
        \item Un nodo intermedio X con una route valida per D in cache può inviare un \texttt{RREP} per conto di D, velocizzando la scoperta e riducendo il flooding.
    \end{itemize}
    \item \textbf{Route Error (\texttt{RERR}):}
    \begin{itemize}
        \item Se un link si rompe, il nodo a monte invia un \texttt{RERR} verso la sorgente, indicando il link rotto. I nodi aggiornano le cache.
    \end{itemize}
    \item \textbf{Attenzione alle Cache Obsolete (Stale Caches):} Possono degradare le prestazioni.
    \item \textbf{Vantaggi:} Reattivo, route caching, una scoperta può dare molte route.
    \item \textbf{Svantaggi:} Header dati grande, potenziale flood di \texttt{RREQ}, Route Reply Storm, inquinamento da cache obsolete.
\end{itemize}

\subsection{Tecniche per Ridurre il Flooding dei Pacchetti di Controllo}
\subsubsection{Location-Aided Routing (LAR)}
Sfrutta informazioni sulla \textbf{posizione geografica} (es. GPS).
\begin{itemize}
    \item \textbf{Expected Zone:} Regione geografica dove ci si aspetta si trovi la destinazione D.
    \item \textbf{Request Zone:} L'\texttt{RREQ} è propagato solo all'interno di una "Request Zone" che include S e l'Expected Zone di D.
    \item Se fallisce, S riprova con una Request Zone più grande.
\end{itemize}

\begin{figure}[H]
    \centering
    \begin{tikzpicture}[node distance=1.5cm]
        % Expected Zone
        \node[node_style] (X) at (0,0) {X};
        \node[node_style] (Y) at (0.8, -0.5) {Y};
        \draw[draw=accentcolor, thick] (X) -- (Y) node[midway, above, sloped, font=\small\sffamily] {r};
        \draw[draw=accentcolor, fill=nodecolor!50, fill opacity=0.4, thick] (X) circle (1.2cm);
        \node[below=0.1cm of X, xshift=0cm, yshift=-1cm, font=\sffamily\small] {Expected Zone};
        \node[font=\sffamily\small, text width=4cm, anchor=west] at (X.east) [xshift=0.75cm, yshift=0cm]
        {X: ultima pos. nota di D (t0)\\Y: pos. attuale di D (t1, incognita a S)\\r = (t1-t0) * stima velocità D};

        % Request Zone (più a destra)
        \begin{scope}[xshift=6cm]
            \node[node_style] (S_lar) {S};
            \node[node_style, right=1cm of S_lar] (B_lar) {B};
            \node[node_style, above left=0.8cm and 0.5cm of S_lar] (A_lar) {A};

            % Expected Zone (come prima, ma più piccola e dentro Request Zone)
            \node[node_style] (X_rq) at (3,0) {}; \node at (X_rq) {\small X};
            \node[node_style] (Y_rq) at (3.4, -0.25) {}; \node at (Y_rq) {\small Y};
            \draw[draw=accentcolor!70, dashed] (X_rq) -- (Y_rq);
            \draw[draw=accentcolor!70, fill=nodecolor!20, fill opacity=0.3, dashed] (X_rq) circle (0.6cm);

            \node (EZ_label) [below=0.1cm of X_rq, xshift=0cm, yshift=-0.5cm, font=\sffamily\small] {Exp.Zone};

            % Request Zone che contiene S e Expected Zone
            \coordinate (sw_corner) at ($(S_lar.south west)+(-0.5,-0.5)$);
            \coordinate (ne_corner) at ($(X_rq.north east)+(0.8,0.8)$); % Estende oltre EZ
             \node[zone_style, fit=(S_lar) (X_rq) (EZ_label) (sw_corner) (ne_corner)] (RZ) {};
            \node[above right, font=\sffamily\small] at (RZ.north west) {Request Zone};

            \draw[link_style, gray] (S_lar) -- (A_lar);
            \draw[link_style, gray] (S_lar) -- (B_lar);
            \node[info_box, below=of RZ, yshift=1cm, text width=4cm] {Solo nodi nella Request Zone (es. B) inoltrano \texttt{RREQ}. A è fuori.};
        \end{scope}
    \end{tikzpicture}
    \caption{Expected Zone e Request Zone in LAR.}
    \label{fig:lar_zones}
\end{figure}

\begin{itemize}
    \item \textbf{Vantaggi LAR:} Riduce scope del flood \texttt{RREQ}, riduce overhead.
    \item \textbf{Svantaggi LAR:} Nodi devono conoscere posizione, non considera ostacoli fisici.
\end{itemize}

\subsubsection{Altri approcci basati sulla posizione:}
\begin{itemize}
    \item \textbf{DREAM (Distance Routing Effect Algorithm for Mobility):} Pacchetti dati inoltrati in un cono verso la destinazione.
    \item \textbf{GEDIR (Geographic Distance Routing):} Inoltra al vicino più vicino geograficamente alla destinazione. Può fallire con "minimi locali".
\end{itemize}

\subsection{Ad Hoc On-Demand Distance Vector Routing (AODV)}
Protocollo \textbf{reattivo} che combina DSR (on-demand) con distance-vector (tabelle di routing).
\begin{itemize}
    \item \textbf{Obiettivo:} Ridurre overhead eliminando la route dagli header dati.
    \item \textbf{Route Request (\texttt{RREQ}):}
    \begin{itemize}
        \item Simile a DSR, S invia \texttt{RREQ} in flooding.
        \item Un nodo che inoltra un \texttt{RREQ} imposta un \textbf{puntatore di route inversa} verso S.
    \end{itemize}
    \item \textbf{Route Reply (\texttt{RREP}):}
    \begin{itemize}
        \item D (o un intermedio con route "fresca") invia \texttt{RREP}.
        \item Freschezza determinata da \textbf{destination sequence numbers}.
        \item \texttt{RREP} viaggia lungo il \textbf{percorso inverso}.
        \item Mentre \texttt{RREP} torna, ogni nodo crea un \textbf{puntatore di route diretta} verso D.
    \end{itemize}
    \item \textbf{Invio Dati:} Usa entry "next-hop". \textbf{Route non inclusa nell'header.}
    \item \textbf{Mantenimento Route e Timeout:} Entry per route inverse e dirette scadono se non usate.
    \item \textbf{Sommario AODV:} No route in header, tabelle solo per route attive, max un next-hop per dest, route non usate scadono.
\end{itemize}

\subsection{Optimized Link State Routing (OLSR)}
Protocollo \textbf{proattivo} basato su link-state, ottimizzato per MANET.
\begin{itemize}
    \item Riduce l'overhead del flooding usando i \textbf{Multipoint Relays (MPR)}.
    \item Ogni nodo seleziona un sottoinsieme dei suoi vicini a 1-hop come MPR.
    \item Gli MPR sono scelti per raggiungere tutti i vicini a 2-hop.
    \item \textbf{Solo gli MPR} inoltrano i messaggi di controllo (link state) generati da quel nodo.
    \item Le route usano preferenzialmente i nodi MPR come intermedi.
\end{itemize}
\begin{figure}[H]
    \centering
    \begin{tikzpicture}[node distance=1.3cm and 1.8cm, font=\sffamily\small]
        \node[important_node_style] (A) {A};
        \node[node_style, above left=of A] (B) {B};
        \node[node_style, below left=of A] (G) {G};
        \node[important_node_style, left=of A, fill=accentcolor!80!nodecolor] (C) {C}; % MPR di A
        \node[node_style, below=of A] (D) {D};
        \node[important_node_style, right=of A, fill=accentcolor!80!nodecolor] (E) {E}; % MPR di A
        \node[node_style, above right=of A] (F) {F};
        \node[important_node_style, right=of E] (H) {H};
        \node[node_style, above right=of H] (J) {J};
        \node[important_node_style, below right=of H, fill=accentcolor!80!nodecolor] (K) {K}; % MPR di H

        \draw[link_style] (A) -- (B); \draw[link_style] (A) -- (C); \draw[link_style] (A) -- (D);
        \draw[link_style] (A) -- (E); \draw[link_style] (A) -- (F);
        \draw[link_style] (B) -- (C); \draw[link_style] (C) -- (G); \draw[link_style] (D) -- (E);
        \draw[link_style] (E) -- (F); \draw[link_style] (E) -- (H);
        \draw[link_style] (H) -- (J); \draw[link_style] (H) -- (K);

        \node[info_box, below=of D, yshift=0.5cm, xshift=2cm, text width=7cm]
        {Nodo A: MPRs sono C ed E (colorati diversamente). Solo C ed E inoltrano info da A.
        Nodo H: MPRs potrebbero essere E e K. Se E ha già inoltrato info (come MPR di A), solo K la inoltra per H.};
    \end{tikzpicture}
    \caption{Esempio di Multipoint Relays in OLSR.}
    \label{fig:olsr_mpr}
\end{figure}

\section{Mobile IP}
Permette a un dispositivo mobile di mantenere lo stesso indirizzo IP spostandosi tra reti IP (con infrastruttura). \textbf{Non è un protocollo di routing per MANET pure}.
\begin{itemize}
    \item \textbf{Componenti Chiave:} Mobile Node (MN), Home Agent (HA), Foreign Agent (FA), Home Address, Care-of Address (CoA).
    \item \textbf{Funzionamento:}
    \begin{enumerate}
        \item MN si sposta in una rete visitata, ottiene un CoA dal FA.
        \item FA informa HA del CoA del MN.
        \item Pacchetti per l'Home Address del MN arrivano all'HA.
        \item L'HA \textbf{tunnelizza} i pacchetti al CoA del MN (verso FA).
        \item FA de-incapsula e consegna al MN.
    \end{enumerate}
    \item \textbf{Problema del "Triangle Routing":} Pacchetti dal corrispondente al MN passano sempre per HA.
\end{itemize}

\begin{figure}[H]
    \centering
    \begin{tikzpicture}[node distance=2cm and 2.5cm, font=\sffamily\small]
        \node[rectangle, minimum width=5cm, minimum height=3cm, fill=bg_custom, draw=accentcolor] (Internet) {Internet};
        \node[node_style, rectangle, fill=accentcolor!70] (HA) [left=1cm of Internet, yshift=0.5cm] {HA (R1)};
        \node[node_style, rectangle, fill=accentcolor!70] (FA2) [right=1cm of Internet, yshift=0.8cm] {FA2 (R2)};
        \node[node_style, rectangle, fill=accentcolor!70] (FA3) [right=1cm of Internet, yshift=-0.8cm] {FA3 (R3)};
        \node[node_style, fill=highlightcolor!60!nodecolor, text=black] (MN_home) [below left=1cm and 0.2cm of HA] {MN (X)};
        \node[node_style, fill=highlightcolor!60!nodecolor, text=black] (MN_FA2) [right=0.5cm of FA2] {MN (X)};
        \node[node_style, fill=highlightcolor!60!nodecolor, text=black] (MN_FA3) [right=0.5cm of FA3] {MN (X)};

        % Collegamenti
        \draw[link_style] (HA) -- (Internet.west);
        \draw[link_style] (FA2) -- (Internet.east);
        \draw[link_style] (FA3) -- (Internet.east);
        \draw[link_style,dashed] (MN_home) -- (HA);
        \draw[link_style,dashed] (MN_FA2) -- (FA2);
        \draw[link_style,dashed] (MN_FA3) -- (FA3);

        % Flusso tunnel
        \coordinate (Sender) at ($(Internet.north)+(-1,0.5)$);
        \node[node_style, ellipse, fill=bg_custom!70] (S) at (Sender) {Sender};
        \draw[highlight_link_style, bend left=10] (S) to node[above,sloped,font=\small] {1. To Home Addr} (HA);
        \draw[highlight_link_style, bend left=10, dotted] (HA) to node[above,sloped,font=\small] {2. Tunnel to CoA} (FA2);
        \draw[highlight_link_style, bend left=10, dotted] (FA2) to node[above,sloped,font=\small] {3. Deliver} (MN_FA2);
    \end{tikzpicture}
    \caption{Funzionamento semplificato di Mobile IP.}
    \label{fig:mobile_ip}
\end{figure}


\section{TCP su Mobile Ad Hoc Networks}
Il Transmission Control Protocol (TCP) fornisce trasporto affidabile e orientato alla connessione.

\subsection{Basi di IP e TCP}
\begin{itemize}
    \item \textbf{IP:} Consegna "best-effort" (pacchetti persi, fuori ordine, duplicati).
    \item \textbf{TCP:}
    \begin{itemize}
        \item Consegna affidabile e ordinata.
        \item Controllo di congestione e flusso.
        \item Affidabilità tramite ritrasmissioni.
        \item Semantica end-to-end.
        \item \textbf{ACK Cumulativi:} \texttt{ack n} conferma byte fino a \texttt{n-1}. Un nuovo ACK cumulativo solo per pacchetti in sequenza.
        \item \textbf{Duplicate Acknowledgements (DupACKs):} Generato per pacchetti fuori ordine. È un ACK per l'ultimo byte \textit{in sequenza} ricevuto.
    \end{itemize}
\end{itemize}

\subsection{Controllo di Flusso Basato su Finestra (Sliding Window) in TCP}
TCP utilizza un protocollo a finestra scorrevole per il controllo di flusso e congestione.
\begin{itemize}
    \item \textbf{Finestra Effettiva di Invio:} Il mittente può inviare una quantità di dati non ancora confermati pari al minimo tra la finestra pubblicizzata dal ricevitore (\texttt{rwnd}, receiver's advertised window) e la finestra di congestione (\texttt{cwnd}, congestion window) calcolata dal mittente.
        \[ \text{Finestra Effettiva (byte)} = \min(\texttt{rwnd}, \texttt{cwnd}) \]
    \item \texttt{rwnd}: Determinata dallo spazio buffer disponibile al ricevitore. Comunica al mittente quanti dati può accettare.
    \item \texttt{cwnd}: Stima del mittente della capacità disponibile lungo il percorso di rete. Adattata dinamicamente in base al feedback dalla rete (ACK, perdite).
\end{itemize}

\subsubsection{Dimensione Ideale della Finestra e BDP}
Come menzionato precedentemente, il BDP (Prodotto Banda-Ritardo) è cruciale:
\[ \text{BDP (bit)} = \text{Bitrate Canale (bps)} \times \text{RTT (s)} \]
\begin{itemize}
    \item \textbf{Dimensione Ideale Finestra $\approx$ BDP:} Per mantenere la "pipeline" di rete piena e massimizzare il throughput, la finestra effettiva di invio dovrebbe essere vicina al BDP.
    \item \textbf{Se Finestra Effettiva $<$ BDP:} Il mittente non invia abbastanza dati per riempire la capacità della rete durante un RTT, portando a un throughput sub-ottimale (banda sprecata).
    \item \textbf{Se Finestra Effettiva $>$ BDP:} Il mittente invia più dati di quanti la rete possa gestire in un RTT. Questo porta all'accumulo di pacchetti nei buffer dei router intermedi, aumentando l'RTT (a causa dei ritardi di accodamento) e incrementando la probabilità di perdita di pacchetti per overflow dei buffer, che a sua volta scatena i meccanismi di controllo della congestione di TCP.
\end{itemize}

\subsection{Rilevamento Perdita Pacchetti}
\begin{enumerate}
    \item \textbf{Retransmission Timeout (RTO):}
    \begin{itemize}
        \item Timer per il pacchetto più vecchio non confermato.
        \item Se scade, pacchetto assunto perso e ritrasmesso. RTO calcolato dinamicamente.
    \end{itemize}
    \item \textbf{Fast Retransmit:}
    \begin{itemize}
        \item Se il mittente riceve \textbf{tre DupACKs consecutivi}, assume perdita e ritrasmette subito il pacchetto successivo a quello ackato, senza aspettare RTO.
    \end{itemize}
\end{enumerate}

\subsection{Controllo e Prevenzione della Congestione}
TCP assume perdita = congestione.
\begin{enumerate}
    \item \textbf{Slow Start:}
    \begin{itemize}
        \item \texttt{cwnd} = 1 MSS.
        \item \texttt{cwnd} cresce esponenzialmente (raddoppia ogni RTT) fino a \texttt{slow-start threshold (ssthresh)}.
    \end{itemize}
    \item \textbf{Congestion Avoidance:}
    \begin{itemize}
        \item Quando \texttt{cwnd} $\ge$ \texttt{ssthresh}, \texttt{cwnd} cresce linearmente (+1 MSS per RTT).
    \end{itemize}
    \item \textbf{Reazione alla Perdita:} TCP interpreta la perdita di pacchetti come un segnale di congestione.
    \begin{itemize}
        \item \textbf{Timeout (RTO):} Indica una congestione severa o una perdita prolungata.
        \begin{itemize}
            \item La soglia di slow start (\texttt{ssthresh}) viene impostata a:
                \[ \texttt{ssthresh} = \max\left(\frac{\min(\texttt{cwnd}_{\text{attuale}}, \texttt{rwnd})}{2}, 2 \times \text{MSS}\right) \]
                (Dove MSS è la Maximum Segment Size). In molte implementazioni, si semplifica a $\texttt{cwnd}_{\text{attuale}}/2$.
            \item La finestra di congestione (\texttt{cwnd}) viene resettata a $1 \times \text{MSS}$.
            \item TCP rientra nella fase di \textbf{Slow Start}.
        \end{itemize}
        \item \textbf{3 DupACKs (Fast Retransmit e Fast Recovery):} Indica una perdita isolata, probabilmente con la rete ancora capace di trasportare traffico.
        \begin{itemize}
            \item \texttt{ssthresh} viene impostata come nel caso di timeout (spesso $\texttt{cwnd}_{\text{attuale}}/2$).
            \item \texttt{cwnd} viene impostata a $\texttt{ssthresh}$ (o $\texttt{ssthresh} + 3 \times \text{MSS}$ in alcune varianti come Reno, per tener conto dei pacchetti che hanno generato i DupACK e che hanno lasciato la rete).
            \item TCP entra (o rimane) direttamente in \textbf{Congestion Avoidance}, evitando la drastica riduzione di Slow Start.
        \end{itemize}
    \end{itemize}
\end{enumerate}

\begin{figure}[H]
    \centering
    \includegraphics[width=1\textwidth]{images/cwnd_timeout.png}
    \caption{Effetto di un Timeout su CWND (Slow Start, Congestion Avoidance, Timeout, di nuovo Slow Start e Congestion Avoidance).}
    \label{fig:cwnd_timeout}
\end{figure}

\begin{figure}[H]
    \centering
    \includegraphics[width=1\textwidth]{images/cwnd_fastrecovery.png}
    \caption{Effetto di Fast Recovery su CWND (CWND si dimezza e continua in Congestion Avoidance).}
    \label{fig:cwnd_fastrecovery}
\end{figure}


\subsection{Sfide di TCP su MANET}
\begin{itemize}
    \item \textbf{Perdita di Pacchetti non dovuta a Congestione:} Errori wireless, collisioni, \textbf{rottura di route per mobilità}. TCP reagisce riducendo \texttt{cwnd} inutilmente.
    \item \textbf{Variazioni dell'RTT:} Cambi di route rendono difficile calcolare RTO accurato.
    \item \textbf{Out-of-Order Delivery:} Causa DupACKs e Fast Retransmit inutili.
    \item \textbf{Impatto Cache Obsolete (es. DSR):} Perdite su route obsolete, TCP reagisce male, lento recupero.
\end{itemize}
Molti lavori cercano di adattare TCP per MANET.

\subsection{Protocolli di Trasporto Fondamentali e Loro Prestazioni (da sapere per esame!!!)}
\subsubsection{Prestazioni di Stop & Wait (S&W)}
Il protocollo Stop & Wait è il più semplice meccanismo di controllo di flusso e errore. Il mittente invia un pacchetto e attende l'Acknowledgement (ACK) prima di inviare il successivo.

\begin{itemize}
    \item \textbf{Tempo di Trasmissione di un pacchetto ($T_{tx}$):} È il tempo necessario per immettere tutti i bit del pacchetto sul canale.
        \[ T_{tx} = \frac{\text{Dimensione Pacchetto (bit)}}{\text{Bitrate Canale (bps)}} \]
        Ad esempio, per un pacchetto di 1000 Byte (8000 bit) su un canale da 1 Gbps ($10^9$ bps):
        \[ T_{tx} = \frac{8000 \text{ bit}}{10^9 \text{ bps}} = 8 \times 10^{-6} \text{ s} = 8 \text{ µs} \]

    \item \textbf{Round Trip Time (RTT):} È il tempo che intercorre tra l'invio di un pacchetto e la ricezione del suo ACK, escludendo il tempo di trasmissione dell'ACK stesso se piccolo. Comprende i tempi di propagazione e di accodamento.

    \item \textbf{Utilizzo del Canale (Channel Utilization, U):} È la frazione di tempo in cui il mittente è effettivamente impegnato a trasmettere dati. In S&W, il ciclo completo per un pacchetto è $T_{tx} + RTT$ (tempo per trasmettere + tempo per attendere ACK).
        \[ U = \frac{T_{tx}}{T_{tx} + RTT} \]
        Riprendendo l'esempio precedente, con RTT = 30 ms (30000 µs):
        \[ U = \frac{8 \text{ µs}}{8 \text{ µs} + 30000 \text{ µs}} = \frac{8}{30008} \approx 0.000266 \text{ (cioè 0.0266\%)} \]
        Questo dimostra la bassa efficienza di S&W su canali con alto prodotto banda-ritardo.

    \item \textbf{Throughput Effettivo ($Th_{eff}$):} È la quantità di dati utili trasferiti nell'unità di tempo.
        \[ Th_{eff} = \frac{\text{Dimensione Pacchetto (bit)}}{T_{tx} + RTT} \]
        Con i dati dell'esempio:
        \[ Th_{eff} = \frac{8000 \text{ bit}}{30008 \text{ µs}} \approx 266.59 \text{ Kbps} \]
        \textit{Nota per esercizi:} A volte, per semplificazione o in contesti specifici, il throughput può essere calcolato come $\frac{\text{Dimensione Pacchetto}}{\text{RTT}}$, specialmente se $T_{tx} \ll RTT$ o se si intende la quantità di dati inviata per ciclo RTT idealizzato. Tuttavia, la formula completa $L/(T_{tx}+RTT)$ è più generale per S&W. L'utilizzo percentuale del canale richiede sempre la conoscenza del bitrate del canale (capacità) per calcolare $T_{tx}$.
\end{itemize}

\subsubsection{Prodotto Banda-Ritardo (Bandwidth-Delay Product - BDP)}
Il prodotto banda-ritardo (BDP) rappresenta la quantità di dati che possono essere "in transito" sulla rete, ossia riempire la "pipeline" tra mittente e ricevitore.
\[ \text{BDP (bit)} = \text{Bitrate Canale (bps)} \times \text{RTT (s)} \]
Idealmente, per massimizzare il throughput, si dovrebbe inviare una quantità di dati pari al BDP prima di attendere un ACK. Se si invia meno del BDP (come in Stop & Wait se $L < \text{BDP}$), il canale rimane sottoutilizzato. Se si invia molto più del BDP senza adeguato controllo, si può causare congestione.

\subsubsection{Pipelining: Go-Back-N (GBN) e Selective Repeat (SR)}
Per superare i limiti di Stop & Wait e sfruttare il BDP, si usano protocolli a pipeline che consentono al mittente di inviare più pacchetti ("finestra" di pacchetti) prima di ricevere un ACK.
\begin{itemize}
    \item \textbf{Go-Back-N (GBN):} Permette una finestra di trasmissione $N$. In caso di perdita del pacchetto $i$, il ricevitore scarta tutti i pacchetti successivi e il mittente ritrasmette da $i$ in poi.
        \begin{itemize}
            \item \textbf{Dimensione della Finestra (N):} Per un controllo di flusso e congestione efficace, $N$ (in termini di dati) non dovrebbe superare il minimo tra la capacità del buffer del ricevitore (advertised window) e la capacità stimata della rete (congestion window). In termini di pacchetti, $N$ deve essere tale che $N \times \text{DimensionePacchetto} \le \text{BDP}$ per un utilizzo ottimale senza eccessivo accodamento iniziale.
            Praticamente, $N$ è spesso limitato da:
            \[ N_{\text{pacchetti}} \approx \min\left(\frac{\text{Buffer Ricevitore}}{\text{Dimensione Pacchetto}}, \frac{\text{Capacità Router Lento} \times \text{RTT}}{\text{Dimensione Pacchetto}}\right) \]
            (La seconda parte è una stima della capacità della rete).
        \end{itemize}
    \item \textbf{Selective Repeat (SR):} Più efficiente di GBN, permette al ricevitore di bufferizzare pacchetti fuori ordine e al mittente di ritrasmettere selettivamente solo i pacchetti persi.
\end{itemize}

\end{document}