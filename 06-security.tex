%%%%%%%%%%%%%%%%%%%%%%%%%%%%%%%%%%%%%%%%%%%%%%%%%%%%%%%%%%%%%%%%%%%%%%%%%%%%%%%
% PREAMBOLO COMUNE PER APPUNTI (Stile Scuro)
%
% Questo file contiene tutte le impostazioni e i pacchetti comuni.
% NON contiene \begin{document} o \end{document}.
%
% Istruzioni per la compilazione del file principale:
% pdflatex -shell-escape nomefile_principale.tex
%%%%%%%%%%%%%%%%%%%%%%%%%%%%%%%%%%%%%%%%%%%%%%%%%%%%%%%%%%%%%%%%%%%%%%%%%%%%%%%

\documentclass{article}

% --- Encoding e lingua ---
\usepackage[utf8]{inputenc}
\usepackage[italian]{babel}

% --- Margini e layout ---
\usepackage{geometry}
\geometry{a4paper, margin=1in}

% --- Font sans-serif (come Helvetica) ---
\usepackage[scaled]{helvet}
\renewcommand{\familydefault}{\sfdefault}
\usepackage[T1]{fontenc}

% --- Matematica ---
\usepackage{amsmath}
\usepackage{amssymb}

% --- Liste personalizzate ---
\usepackage{enumitem}
% \setlist{nosep}

% --- Immagini e Grafica ---
\usepackage{float}
% \usepackage{graphicx}
\usepackage{tikz}
\usetikzlibrary{shapes.geometric, positioning, arrows.meta, calc, fit, backgrounds, patterns, decorations.pathreplacing}

% --- Tabelle Avanzate ---
\usepackage{array}
\usepackage{booktabs}
\usepackage{longtable}

% --- Hyperlink e Metadati PDF ---
\usepackage{hyperref}

\hypersetup{
    colorlinks=true,
    linkcolor=white,
    filecolor=magenta,
    urlcolor=cyan,
    citecolor=green,
    % pdftitle, pdfauthor, ecc. verranno impostati nel file principale
    pdfpagemode=FullScreen,
    bookmarksopen=true,
    bookmarksnumbered=true
}

% --- Licenza del documento ---
\usepackage[
  type={CC},
  modifier={by-sa},
  version={4.0},
]{doclicense}

% --- Colori e Sfondo Nero ---
\usepackage{xcolor}
\pagecolor{black}
\color{white}

% --- Evidenziazione del Codice ---
\usepackage{minted}
\setminted{
    frame=lines,
    framesep=2mm,
    fontsize=\small,
    breaklines=true,
    style=monokai,
    bgcolor=black!80
}
\usemintedstyle{monokai}

% --- Comandi personalizzati per algebra relazionale ---
\newcommand{\Rel}[1]{\textit{#1}} % Per i nomi delle relazioni
\newcommand{\Attr}[1]{\textsf{#1}} % Per i nomi degli attributi

\newcommand{\myunion}{\cup}
\newcommand{\myintersection}{\cap}
\newcommand{\mydifference}{-}
\newcommand{\myrename}[2]{\rho_{#1}(#2)}
\newcommand{\myselectop}[2]{\sigma_{#1}(#2)}
\newcommand{\myproject}[2]{\pi_{#1}(#2)}
\newcommand{\mycartesian}{\times}
\newcommand{\mynaturaljoin}{\bowtie} % Usare \Join da amssymb se disponibile e preferito
\newcommand{\mythetajoin}[3]{#1 \bowtie_{#2} #3} % R1 \bowtie_cond R2

% --- Comandi personalizzati per logica ---
\newcommand{\mylandop}{\wedge}
\newcommand{\myvel}{\vee}
\newcommand{\mynegop}{\neg}
\newcommand{\myforallop}{\forall}
\newcommand{\myexistsop}{\exists}

% --- Join esterni (outer join) ---
% Definizione standard per i join esterni
\def\ojoin{\setbox0=\hbox{$\mynaturaljoin$}%
	\rule[-.02ex]{.25em}{.4pt}\llap{\rule[\ht0]{.25em}{.4pt}}}
\newcommand{\myleftouterjoin}{\mathbin{\ojoin\mkern-5.8mu\mynaturaljoin}}
\newcommand{\myrightouterjoin}{\mathbin{\mynaturaljoin\mkern-5.8mu\ojoin}}
\newcommand{\myfullouterjoin}{\mathbin{\ojoin\mkern-5.8mu\mynaturaljoin\mkern-5.8mu\ojoin}}



\hypersetup{
    pdftitle={Appunti su Sicurezza nelle Reti di Calcolatori},
}

% --- Titolo ---
\title{Appunti su Sicurezza nelle Reti di Calcolatori \\ (Capitolo 8)}
\author{Basato sulle slide del Prof.} % Nome del prof. omesso come da richiesta
\date{\today}
\begin{document}

\maketitle
\tableofcontents
\newpage

\section{Cos'è la Sicurezza di Rete?}
\label{sec:cos_e_sicurezza}

La sicurezza di rete si basa su alcuni pilastri fondamentali:

\begin{enumerate}[label=\arabic*.]
    \item \textbf{Riservatezza (Confidentiality):}
    \begin{itemize}
        \item Solo il mittente e il destinatario intenzionale dovrebbero essere in grado di \textit{comprendere} il contenuto del messaggio.
        \item Si ottiene tramite:
        \begin{itemize}
            \item Il mittente \textbf{crittografa} il messaggio.
            \item Il destinatario \textbf{decrittografa} il messaggio.
        \end{itemize}
        \item \textit{Esempio pratico:} Quando inserisci la password per l'home banking, vuoi che solo la banca possa leggerla, non un malintenzionato che intercetta la comunicazione.
    \end{itemize}

    \item \textbf{Autenticazione (Authentication):}
    \begin{itemize}
        \item Sia il mittente che il destinatario vogliono confermare l'identità l'uno dell'altro.
        \item \textit{Esempio pratico:} La banca vuole essere sicura che sei tu a fare login, e tu vuoi essere sicuro di star parlando con il vero sito della banca e non con un sito fasullo.
    \end{itemize}

    \item \textbf{Integrità del Messaggio (Message Integrity):}
    \begin{itemize}
        \item Mittente e destinatario vogliono assicurarsi che il messaggio non sia stato alterato (durante il transito o successivamente) senza che ciò venga rilevato.
        \item \textit{Esempio pratico:} Se invii un bonifico da 100€, vuoi che la banca riceva 100€ e non 10.000€ perché qualcuno ha modificato l'importo.
    \end{itemize}

    \item \textbf{Accesso e Disponibilità (Access and Availability):}
    \begin{itemize}
        \item I servizi devono essere accessibili e disponibili per gli utenti legittimi.
        \item \textit{Esempio pratico:} Il sito della tua università deve essere raggiungibile quando hai bisogno di consultare gli orari delle lezioni, e non reso irraggiungibile da un attacco.
    \end{itemize}
\end{enumerate}

\subsection{Personaggi Comuni in Sicurezza}
\begin{itemize}
    \item \textbf{Alice e Bob:} Due entità (persone, sistemi) che vogliono comunicare in modo sicuro.
    \item \textbf{Trudy (Intruder):} L'antagonista che cerca di intercettare, cancellare o aggiungere messaggi.
\end{itemize}

\begin{figure}[H]
    \centering
    \begin{tikzpicture}[node distance=2cm and 4cm, auto]
        \node (alice) [actor] {Alice};
        \node (bob) [actor, right=of alice] {Bob};
        \node (trudy) [intruder, below = 0.7cm of alice, xshift=2cm] {Trudy};

        \draw[message_r] (alice.east) -- node[midway, above, sloped, text=black!40!white] {canale dati/controllo} (bob.west);
        \draw[->, red, thick, dashed] (trudy.north) |- ($(alice.east)!0.5!(bob.west)$) node[right, text=red] {intercetta};
    \end{tikzpicture}
    \caption{Scenario base: Alice, Bob e l'intruso Trudy.}
\end{figure}

\subsection{Chi sono Alice e Bob nella vita reale?}
\begin{itemize}
    \item Persone che si scambiano email.
    \item Browser web e server web durante transazioni elettroniche (es. acquisti online).
    \item Client e server di online banking.
    \item Server DNS.
    \item Router che si scambiano aggiornamenti delle tabelle di routing.
\end{itemize}

\subsection{Cosa può fare un \textit{cattivo} (Trudy)?}
\begin{itemize}
    \item \textbf{Eavesdrop (Origliare):} Intercettare messaggi.
    \item \textbf{Insert (Inserire attivamente):} Introdurre messaggi falsi nella connessione.
    \item \textbf{Impersonation (Impersonificazione/Spoofing):} Fingere di essere qualcun altro, ad es. falsificando l'indirizzo IP sorgente in un pacchetto.
    \item \textbf{Hijacking (Dirottamento):} \textquotedblleft Prendere il controllo\textquotedblright{} di una connessione esistente, sostituendosi al mittente o al destinatario.
    \item \textbf{Denial of Service (DoS - Negazione del Servizio):} Impedire che un servizio venga utilizzato da altri, ad es. sovraccaricando le risorse del server.
\end{itemize}

\section{Principi di Crittografia}
\label{sec:principi_crittografia}

\subsection{Il Linguaggio della Crittografia}
\begin{itemize}
    \item $m$ (\textbf{plaintext}): Il messaggio originale, in chiaro.
    \item \textbf{Algoritmo di Crittografia (Encryption Algorithm):} Processo per trasformare il plaintext.
    \item \textbf{Chiave di Crittografia ($K_A$):} Un valore segreto usato dall'algoritmo per crittografare.
    \item $K_A(m)$ (\textbf{ciphertext}): Il messaggio crittografato, illeggibile senza la chiave.
    \item \textbf{Algoritmo di Decrittografia (Decryption Algorithm):} Processo per ritrasformare il ciphertext in plaintext.
    \item \textbf{Chiave di Decrittografia ($K_B$):} Un valore segreto usato dall'algoritmo per decrittografare.
    \item Se le chiavi $K_A$ e $K_B$ sono le stesse (o facilmente derivabili l'una dall'altra), si parla di \textbf{crittografia simmetrica}. Se sono diverse e non derivabili, si parla di \textbf{crittografia asimmetrica (o a chiave pubblica)}.
\end{itemize}

\begin{figure}[H]
    \centering
    \begin{tikzpicture}[node distance=1.5cm and 2.5cm, auto]
        \node (plaintext1) [block] {plaintext ($m$)};
        \node (keyA) [key, above left=0.3cm and 0.2cm of plaintext1, label=left:Alice's Key] {$K_A$};
        \node (enc_algo) [process, right=of plaintext1] {Encryption Algorithm};
        \node (ciphertext) [block, right=of enc_algo] {ciphertext $K_A(m)$};
        \node (keyB) [key, above right=0.3cm and 0.2cm of ciphertext, label=right:Bob's Key] {$K_B$};
        \node (dec_algo) [process, right=of ciphertext] {Decryption Algorithm};
        \node (plaintext2) [block, right=of dec_algo] {plaintext ($m$)};
        \node (trudy_crypto) [intruder, below=of ciphertext] {Trudy};

        \draw[dataflow] (plaintext1) -- (enc_algo);
        \draw[dataflow] (keyA) |- (enc_algo);
        \draw[dataflow] (enc_algo) -- (ciphertext);
        \draw[dataflow] (ciphertext) -- (dec_algo);
        \draw[dataflow] (keyB) |- (dec_algo);
        \draw[dataflow] (dec_algo) -- (plaintext2);
        \draw[->, red, dashed] (trudy_crypto.north) -- (ciphertext);
    \end{tikzpicture}
    \caption{Flusso base della crittografia.}
\end{figure}

\subsection{Rompere uno Schema di Crittografia (Attacchi)}
\begin{enumerate}
    \item \textbf{Ciphertext-only attack (Attacco al solo ciphertext):} Trudy ha solo il ciphertext e cerca di decifrarlo.
    \begin{itemize}
        \item \textbf{Brute force:} Provare tutte le possibili chiavi.
        \item \textbf{Analisi statistica:} Analizzare le frequenze delle lettere/simboli.
    \end{itemize}
    \item \textbf{Known-plaintext attack (Attacco con plaintext noto):} Trudy ha accesso a coppie di plaintext e al corrispondente ciphertext.
    \item \textbf{Chosen-plaintext attack (Attacco con plaintext scelto):} Trudy può scegliere dei plaintext, farli crittografare dal sistema e ottenere i ciphertext corrispondenti.
\end{enumerate}

\subsection{Crittografia a Chiave Simmetrica}
\begin{itemize}
    \item Alice e Bob condividono la \textbf{stessa chiave segreta ($K_S$)}.
    \item \textit{Problema:} Come fanno Alice e Bob a concordare la chiave $K_S$ in modo sicuro?
\end{itemize}

\subsubsection{Cifrario a Sostituzione Semplice (Monoalfabetico)}
\begin{itemize}
    \item Ogni lettera del plaintext è sostituita da un'altra lettera.
    \item \textit{Esempio plaintext:} \texttt{abcdefghijklmnopqrstuvwxyz}
    \item \textit{Esempio ciphertext:} \texttt{mnbvcxzasdfghjklpoiuytrewq}
    \item La \textbf{chiave di crittografia} è la mappatura. Debole all'analisi statistica.
\end{itemize}

\subsubsection{DES (Data Encryption Standard)}
\begin{itemize}
    \item Standard USA (NIST 1993), ora obsoleto per molte applicazioni.
    \item Chiave simmetrica a \textbf{56 bit}, input plaintext a 64 bit.
    \item \textbf{Block cipher} con \textbf{cipher block chaining (CBC)}.
    \item \textbf{Sicurezza:} Una chiave a 56 bit può essere trovata con brute force in meno di un giorno.
    \item \textbf{3DES (Triple DES):} Crittografa 3 volte con 3 chiavi diverse. Più sicuro ma più lento.
    \item \textbf{Funzionamento DES (generale):} Permutazione iniziale, 16 \textit{round} identici con sotto-chiavi diverse, permutazione finale.
\end{itemize}

\subsubsection{AES (Advanced Encryption Standard)}
\begin{itemize}
    \item Standard NIST che ha sostituito DES (Novembre 2001).
    \item Processa dati in blocchi da \textbf{128 bit}.
    \item Chiavi simmetriche da \textbf{128, 192, o 256 bit}.
    \item \textbf{Sicurezza:} Considerato molto sicuro. Un attacco brute force su AES (128 bit) impiegherebbe trilioni di anni.
\end{itemize}

\subsection{Crittografia a Chiave Pubblica (Asimmetrica)}
\begin{itemize}
    \item Mittente e destinatario \textbf{non} condividono una chiave segreta.
    \item Ogni utente ha una coppia di chiavi:
    \begin{itemize}
        \item \textbf{Chiave Pubblica ($K_B^+$):} Conosciuta da tutti. Usata per crittografare messaggi destinati a Bob o per verificare firme digitali di Bob.
        \item \textbf{Chiave Privata ($K_B^-$):} Conosciuta solo dal proprietario (Bob). Usata per decrittografare messaggi crittografati con la sua chiave pubblica o per firmare digitalmente messaggi.
    \end{itemize}
\end{itemize}
\textit{Flusso (Alice invia a Bob):} plaintext ($m$) $\xrightarrow{K_B^+}$ ciphertext ($K_B^+(m)$) $\xrightarrow{K_B^-}$ plaintext ($m$)

\subsubsection{RSA (Rivest, Shamir, Adelson Algorithm)}
\begin{itemize}
    \item L'algoritmo a chiave pubblica più diffuso.
    \item \textbf{Prerequisito: Aritmetica Modulare}
    \begin{itemize}
        \item $x \pmod n$: resto della divisione di $x$ per $n$.
        \item Proprietà: $(a \pmod n)^d \pmod n = a^d \pmod n$.
    \end{itemize}
    \item \textbf{Creazione della Coppia di Chiavi Pubblica/Privata RSA:}
    \begin{enumerate}
        \item Scegliere due numeri primi grandi, $p$ e $q$.
        \item Calcolare $n = p \cdot q$ e $z = (p-1)(q-1)$.
        \item Scegliere $e$ (con $e<n$) tale che $e$ e $z$ siano relativamente primi.
        \item Scegliere $d$ tale che $e \cdot d \equiv 1 \pmod z$.
        \item \textbf{Chiave Pubblica:} $K_B^+ = (n, e)$
        \item \textbf{Chiave Privata:} $K_B^- = (n, d)$
    \end{enumerate}
    \item \textbf{Crittografia e Decrittografia RSA:}
    \begin{itemize}
        \item Crittografare $m$ ($m < n$): $c = m^e \pmod n$
        \item Decrittografare $c$: $m = c^d \pmod n$
        \item Quindi: $m = (m^e \pmod n)^d \pmod n$
    \end{itemize}
    \item \textbf{Altra Proprietà Importante di RSA (Commutatività):}
    $K_B^-(K_B^+(m)) = m$ e $K_B^+(K_B^-(m)) = m$. La seconda forma è usata per le firme digitali.
    \item \textbf{Perché RSA è sicuro?} Si basa sulla difficoltà computazionale di \textbf{fattorizzare numeri molto grandi} ($n$).
    \item \textbf{RSA in Pratica: Chiavi di Sessione (Session Keys):}
    RSA è lento. Si usa RSA per scambiare una \textbf{chiave simmetrica di sessione ($K_S$)}, e poi si usa AES/DES (più veloci) con $K_S$ per i dati.
\end{itemize}

\section{Integrità dei Messaggi e Autenticazione}
\label{sec:integrita_autenticazione}

\subsection{Autenticazione}
\textbf{Obiettivo:} Bob vuole che Alice \textquotedblleft provi\textquotedblright{} la sua identità.

\begin{itemize}
    \item \textbf{Protocollo ap1.0:} Alice: \textquotedblleft Sono Alice\textquotedblright{}. \textit{Fallimento:} Trudy può dirlo.
    \item \textbf{Protocollo ap2.0:} Alice: \textquotedblleft Sono Alice\textquotedblright{} (con IP di Alice). \textit{Fallimento:} Trudy può fare IP spoofing.
    \item \textbf{Protocollo ap3.0:} Alice: \textquotedblleft Sono Alice\textquotedblright{} + password. \textit{Fallimento:} Playback attack.
    \item \textbf{Protocollo ap3.1:} Alice: \textquotedblleft Sono Alice\textquotedblright{} + password crittografata. \textit{Fallimento:} Playback attack funziona ancora.
    \item \textbf{Protocollo ap4.0: Autenticazione con Nonce (per evitare playback):}
    \begin{itemize}
        \item \textbf{Nonce (R):} Numero usato una sola volta.
        \item Passaggi:
        \begin{enumerate}
            \item Alice $\rightarrow$ Bob: \textquotedblleft Sono Alice\textquotedblright{}.
            \item Bob $\rightarrow$ Alice: $R$ (nonce).
            \item Alice $\rightarrow$ Bob: $K_{A-B}(R)$ (nonce crittografato con chiave condivisa).
        \end{enumerate}
        \item \textit{Problema:} Richiede una chiave simmetrica $K_{A-B}$ condivisa.
    \end{itemize}
    \item \textbf{Protocollo ap5.0: Autenticazione con Nonce e Chiave Pubblica:}
    \begin{enumerate}
        \item Alice $\rightarrow$ Bob: \textquotedblleft Sono Alice\textquotedblright{}.
        \item Bob $\rightarrow$ Alice: $R$ (nonce).
        \item (Alice chiede la chiave pubblica di Bob, $K_A^+$).
        \item Alice $\rightarrow$ Bob: $K_A^-(R)$ (nonce crittografato con la chiave \textit{privata} di Alice).
        \item Bob verifica con $K_A^+(K_A^-(R)) \stackrel{?}{=} R$.
    \end{enumerate}
    \textit{Buco di Sicurezza ap5.0:} \textbf{Man-in-the-Middle (MITM) Attack}. Trudy si frappone e scambia le sue chiavi pubbliche con Alice e Bob.
\end{itemize}

\subsection{Firme Digitali (Digital Signatures)}
\begin{itemize}
    \item Tecnica crittografica analoga alle firme autografe.
    \item \textbf{Obiettivi:}
    \begin{itemize}
        \item Mittente (Bob) firma, stabilendo proprietà/creazione.
        \item \textbf{Verificabile e Non Falsificabile}.
        \item \textbf{Non Ripudio (Non-repudiation):} Bob non può negare di aver firmato.
    \end{itemize}
    \item \textbf{Firma Digitale Semplice per $m$ (usando RSA):}
    \begin{enumerate}
        \item Bob firma $m$ con la sua \textbf{chiave privata $K_B^-$}: $K_B^-(m)$.
        \item Bob invia ad Alice $(m, K_B^-(m))$.
        \item Alice verifica applicando la \textbf{chiave pubblica di Bob $K_B^+$} alla firma: $K_B^+(K_B^-(m))$.
        \item Se il risultato è $m$, la firma è valida.
    \end{enumerate}
\end{itemize}

\subsection{Digest dei Messaggi (Funzioni Hash Crittografiche)}
Crittografare messaggi lunghi con chiave pubblica è costoso.
\begin{itemize}
    \item \textbf{Obiettivo:} Creare un'\textquotedblleft impronta digitale\textquotedblright{} (\textit{fingerprint}) di lunghezza fissa.
    \item Si applica una \textbf{funzione hash $H$} al messaggio $m$ per ottenere un digest $H(m)$.
    \item \textbf{Proprietà delle Funzioni Hash Crittografiche:}
    \begin{itemize}
        \item Output di lunghezza fissa.
        \item \textbf{Resistenza alla preimmagine (One-way):} Dato $H(m)$, difficile trovare $m$.
        \item \textbf{Resistenza alla seconda preimmagine:} Dato $m_1$, difficile trovare $m_2 \neq m_1$ t.c. $H(m_1) = H(m_2)$.
        \item \textbf{Resistenza alle collisioni:} Difficile trovare $m_1 \neq m_2$ t.c. $H(m_1) = H(m_2)$.
    \end{itemize}
    \item \textbf{Internet Checksum:} NON è una funzione hash crittografica (debole).
    \item \textbf{Firma Digitale con Message Digest:}
    \begin{enumerate}
        \item \textbf{Bob (mittente):}
        \begin{enumerate}
            \item Calcola $H(m)$.
            \item Firma il digest: $K_B^-(H(m))$.
            \item Invia $(m, K_B^-(H(m)))$.
        \end{enumerate}
        \item \textbf{Alice (destinatario):}
        \begin{enumerate}
            \item Riceve $m$ e $S_B = K_B^-(H(m))$.
            \item Calcola $H(m')$ dal messaggio $m'$ ricevuto.
            \item Decrittografa la firma: $H_{firmato} = K_B^+(S_B)$.
            \item Confronta $H(m')$ con $H_{firmato}$. Se uguali, firma valida e messaggio integro.
        \end{enumerate}
    \end{enumerate}
    \item \textbf{Algoritmi di Hash Comuni:} MD5 (insicuro), SHA-1 (debole), SHA-2 (SHA-256, SHA-512 - raccomandati).
\end{itemize}

\subsection{Certificazione della Chiave Pubblica (Public-Key Certification)}
Risolve il problema del MITM per lo scambio di chiavi pubbliche.
\begin{itemize}
    \item \textbf{Autorità di Certificazione (CA - Certification Authority):} Entità fidata che lega una chiave pubblica a un'entità E.
    \item \textbf{Processo:}
    \begin{enumerate}
        \item L'entità E (es. Bob) si registra presso una CA (prova d'identità + chiave pubblica).
        \item La CA crea un \textbf{certificato digitale} contenente: identità di E, chiave pubblica di E, info CA, validità, ecc.
        \item La CA \textbf{firma digitalmente} questo certificato con la \textbf{chiave privata della CA}.
    \end{enumerate}
    \item \textbf{Utilizzo del Certificato:}
    \begin{enumerate}
        \item Alice ottiene il certificato di Bob.
        \item Alice usa la chiave pubblica della CA (che deve conoscere/fidarsi) per verificare la firma sul certificato di Bob.
        \item Se la firma della CA è valida, Alice si fida che la chiave pubblica nel certificato sia di Bob.
    \end{enumerate}
\end{itemize}

\section{Protezione delle E-mail}
\label{sec:protezione_email}

Alice vuole inviare email sicure a Bob (riservatezza, autenticazione, integrità).

\subsection{Solo Riservatezza}
\begin{enumerate}
    \item \textbf{Alice:}
    \begin{enumerate}
        \item Genera chiave simmetrica casuale $K_S$.
        \item Crittografa $m$ con $K_S$: $K_S(m)$.
        \item Crittografa $K_S$ con la chiave pubblica di Bob $K_B^+$: $K_B^+(K_S)$.
        \item Invia $(K_S(m), K_B^+(K_S))$.
    \end{enumerate}
    \item \textbf{Bob:}
    \begin{enumerate}
        \item Usa $K_B^-$ per decrittografare $K_B^+(K_S) \rightarrow K_S$.
        \item Usa $K_S$ per decrittografare $K_S(m) \rightarrow m$.
    \end{enumerate}
\end{enumerate}

\subsection{Solo Autenticazione e Integrità}
\begin{enumerate}
    \item \textbf{Alice:}
    \begin{enumerate}
        \item Calcola $H(m)$.
        \item Firma $H(m)$ con la sua chiave privata $K_A^-$: $K_A^-(H(m))$.
        \item Invia $(m, K_A^-(H(m)))$.
    \end{enumerate}
    \item \textbf{Bob:}
    Verifica la firma usando $K_A^+$ e confrontando $H(m)$.
\end{enumerate}

\subsection{Riservatezza + Autenticazione + Integrità (PGP-like)}
\begin{enumerate}
    \item \textbf{Alice:}
    \begin{enumerate}
        \item Firma $H(m)$ con $K_A^- \rightarrow S_A = K_A^-(H(m))$.
        \item $M' = m \mathbin{\|} S_A$.
        \item Genera $K_S$.
        \item Crittografa $M'$ con $K_S \rightarrow C_1 = K_S(M')$.
        \item Crittografa $K_S$ con $K_B^+ \rightarrow C_2 = K_B^+(K_S)$.
        \item Invia $(C_1, C_2)$.
    \end{enumerate}
    \item \textbf{Bob:}
    Processo inverso per ottenere $m$ e verificare $S_A$.
\end{enumerate}
Alice usa tre chiavi: la sua privata, la pubblica di Bob, e la $K_S$ simmetrica.

\section{Protezione delle Connessioni TCP: SSL/TLS}
\label{sec:ssl_tls}
SSL (Secure Sockets Layer) / TLS (Transport Layer Security).

\begin{itemize}
    \item \textbf{Fornisce:} Riservatezza, Integrità, Autenticazione (server sempre, client opzionale).
    \item \textbf{Posizionamento:} Applicazione $\rightarrow$ \textbf{SSL/TLS} $\rightarrow$ TCP $\rightarrow$ IP.
\end{itemize}

\subsection{Fasi di SSL (Concettuale, \textquotedblleft Toy SSL\textquotedblright{})}
\begin{enumerate}
    \item \textbf{Handshake (Stretta di Mano):}
    \begin{itemize}
        \item Client e Server usano certificati e chiavi pubbliche/private per autenticarsi e scambiarsi un segreto condiviso (\textbf{Master Secret, MS}).
        \item Client genera un \textbf{Pre-Master Secret (PMS)}, lo crittografa con la chiave pubblica del server.
        \item Entrambi calcolano MS da PMS e nonce scambiati.
    \end{itemize}
    \item \textbf{Key Derivation (Derivazione delle Chiavi):}
    \begin{itemize}
        \item Da MS si derivano chiavi di sessione simmetriche (tipicamente 4: 2 per crittografia client $\leftrightarrow$ server, 2 per MAC client $\leftrightarrow$ server).
    \end{itemize}
    \item \textbf{Data Transfer (Trasferimento Dati):}
    \begin{itemize}
        \item Dati suddivisi in \textbf{record}.
        \item Ogni record: (Dati + MAC) crittografati + header SSL (tipo, versione, lunghezza).
        \item Numeri di sequenza impliciti nel MAC per prevenire replay/riordino.
        \item Tipi di record specifici per prevenire Truncation Attack.
    \end{itemize}
    \item \textbf{Connection Closure (Chiusura Connessione):}
    Messaggi speciali (alert \texttt{close\_notify}) per chiudere in modo sicuro.
\end{enumerate}

\begin{figure}[H]
    \centering
    \begin{tikzpicture}[node distance=1cm and 2.5cm]
        \node (client) [actor] {Client (Alice)};
        \node (server) [actor, right=6cm of client] {Server (Bob)};

        % Handshake
        \draw[message_r] (client.east) -- node[midway, above, font=\scriptsize] {1. ClientHello (ciphers, nonce C)} (server.west);
        \draw[message_l] (client.east) -- node[midway, below, font=\scriptsize] {2. ServerHello (cipher choice, cert, nonce S)} (server.west);
        \draw[message_r] (client.east) -- node[midway, above, yshift=0.5cm, font=\scriptsize] {3. ClientKeyExchange ($K_S^+(\text{PMS})$), ChangeCipherSpec, Finished} (server.west);
        \draw[message_l] (client.east) -- node[midway, below, yshift=-0.5cm, font=\scriptsize] {4. ChangeCipherSpec, Finished} (server.west);

        % Data Transfer (Encrypted)
        \node (encrypted_data_label) [below=1.5cm of client, xshift=3cm, font=\small\itshape, text=green!50!white] {Trasferimento Dati Crittografati};
        \draw[<->, green!70!black, thick, dashed] ($(client.south east) + (0,-0.5)$) -- node[midway, below, font=\scriptsize] {Record SSL} ($(server.south west) + (0,-0.5)$);

        % Vertical lines for phases
        \coordinate (midpoint) at ($(client)!0.5!(server)$);
        \draw[dotted, gray] (midpoint) ++(0,1.5cm) -- (midpoint) ++(0,-2.5cm);
        \node[above=2.5cm of $(client)!0.5!(server)$, font=\small\bfseries] {Handshake};
    \end{tikzpicture}
    \caption{Schema semplificato dell'Handshake SSL/TLS.}
\end{figure}


\subsection{Real SSL: Handshake (dettagli)}
\begin{enumerate}
    \item \textbf{ClientHello:} Client invia lista cipher suite, versione SSL/TLS, nonce client.
    \item \textbf{ServerHello, Certificate, ServerHelloDone:} Server sceglie cipher suite, invia certificato, nonce server.
    \item \textbf{ClientKeyExchange, ChangeCipherSpec, EncryptedHandshakeMessage (Finished):}
    Client verifica certificato server, genera PMS, lo crittografa con chiave pubblica server. Entrambi calcolano MS e chiavi di sessione. Client invia MAC di tutti i messaggi handshake finora, crittografato (Finished).
    \item \textbf{Server: ChangeCipherSpec, EncryptedHandshakeMessage (Finished):}
    Server invia il suo messaggio Finished crittografato.
\end{enumerate}
Gli ultimi due messaggi (Finished) proteggono l'handshake da manomissioni. I due nonce prevengono replay di sessioni.

\subsection{SSL Record Protocol}
Dati frammentati $\rightarrow$ Aggiunta MAC $\rightarrow$ Crittografia (frammento+MAC) $\rightarrow$ Header SSL.

\section{Sicurezza a Livello di Rete: IPsec (IP Security)}
\label{sec:ipsec}
Fornisce sicurezza a livello IP (livello 3).

\begin{itemize}
    \item \textbf{\textquotedblleft Blanket coverage\textquotedblright{}:} Protegge tutto il traffico tra due endpoint IPsec.
    \item \textbf{Virtual Private Networks (VPNs):} Traffico crittografato su Internet pubblica.
\end{itemize}

\subsection{Servizi IPsec}
Integrità dati, autenticazione origine, prevenzione replay, confidenzialità.

\subsection{Protocolli IPsec}
\begin{itemize}
    \item \textbf{AH (Authentication Header):} Autenticazione, integrità. \textbf{Non} confidenzialità.
    \item \textbf{ESP (Encapsulating Security Payload):} Autenticazione, integrità, \textbf{confidenzialità}. Più usato.
\end{itemize}

\subsection{Modalità IPsec}
\begin{itemize}
    \item \textbf{Transport Mode:} Protegge payload IP. Header IP originale in gran parte intatto. Host-to-host.
    \item \textbf{Tunnel Mode:} Intero datagramma IP originale incapsulato in nuovo datagramma IP. Tipicamente router-to-router (VPN site-to-site) o client-to-router.
\end{itemize}
\textbf{ESP in Tunnel Mode è la più comune per VPN.}

\subsection{Security Associations (SAs)}
\begin{itemize}
    \item Contratto unidirezionale (simplex) tra due entità IPsec. Contiene parametri (chiavi, algoritmi, SPI).
    \item \textbf{SPI (Security Parameter Index):} Identificatore a 32 bit della SA.
    \item \textbf{SAD (Security Association Database):} Database dove gli endpoint IPsec memorizzano le SA.
\end{itemize}

\subsection{Datagramma IPsec (Tunnel Mode con ESP)}
\texttt{[New IP Hdr] [ESP Hdr] [Original IP Hdr] [Original Payload] [ESP Trailer] [ESP Auth]}
\begin{itemize}
    \item \textbf{Parte Crittografata:} \texttt{[Original IP Hdr] [Original Payload] [ESP Trailer]}
    \item \textbf{Parte Autenticata (MAC):} \texttt{[ESP Hdr] [Parte Crittografata]}
    \item \textbf{Numeri di Sequenza:} Nell'ESP Hdr, per prevenire replay.
\end{itemize}

\begin{figure}[H]
    \centering
    \begin{tikzpicture}[node distance=0mm, scale=0.8, transform shape]
        \small % Smaller font for the diagram
        \node (newip)   [draw, fill=blue!20, minimum height=0.7cm, minimum width=2.5cm, text=white] {New IP Hdr};
        \node (esph)    [draw, fill=green!30, minimum height=0.7cm, minimum width=2.2cm, right=of newip, text=white] {ESP Hdr};
        \node (origip)  [draw, fill=orange!40, minimum height=0.7cm, minimum width=2.5cm, right=of esph, text=white] {Original IP Hdr};
        \node (origpay) [draw, fill=orange!40, minimum height=0.7cm, minimum width=3.0cm, right=of origip, text=white] {Original Payload};
        \node (esptrl)  [draw, fill=orange!40, minimum height=0.7cm, minimum width=2.5cm, right=of origpay, text=white] {ESP Trailer};
        \node (espauth) [draw, fill=purple!50, minimum height=0.7cm, minimum width=2.2cm, right=of esptrl, text=white] {ESP Auth (ICV)};

        % Braces for encrypted and authenticated parts
        \draw [decorate,decoration={brace,amplitude=5pt,mirror},yshift=-0.5cm]
            (origip.south west) -- (esptrl.south east) node [black,midway,yshift=-0.4cm, text=white] {Encrypted};

        \draw [decorate,decoration={brace,amplitude=5pt},yshift=0.5cm]
            (esph.north west) -- (esptrl.north east) node [black,midway,yshift=0.4cm, text=white] {Authenticated (for ICV)};
    \end{tikzpicture}
    \caption{Struttura del datagramma IPsec (ESP Tunnel Mode).}
\end{figure}


\subsection{SPD (Security Policy Database)}
Contiene policy che dicono \textbf{se} e \textbf{quale SA} usare per un dato traffico.
\begin{itemize}
    \item \textbf{SPD:} \textquotedblleft Cosa\textquotedblright{} fare (policy).
    \item \textbf{SAD:} \textquotedblleft Come\textquotedblright{} farlo (parametri SA).
\end{itemize}

\subsection{IKE (Internet Key Exchange)}
Protocollo per negoziare e stabilire dinamicamente le SA IPsec.
\begin{itemize}
    \item \textbf{Autenticazione IKE:}
    \begin{itemize}
        \item \textbf{PSK (Pre-Shared Secret):} Segreto condiviso manualmente.
        \item \textbf{PKI (Public Key Infrastructure):} Chiavi pubbliche/private e certificati.
    \end{itemize}
    \item \textbf{Fasi di IKE:}
    \begin{itemize}
        \item \textbf{Fase 1:} Stabilisce SA IKE bidirezionale (ISAKMP SA). Modalità: Aggressive o Main.
        \item \textbf{Fase 2:} Usa ISAKMP SA per negoziare coppia di SA IPsec.
    \end{itemize}
\end{itemize}

\section{Protezione delle LAN Wireless (Wi-Fi)}
\label{sec:wifi_security}

\subsection{WEP (Wired Equivalent Privacy)}
\textbf{Standard obsoleto e insicuro.}
\begin{itemize}
    \item \textbf{Obiettivi:} Confidenzialità, autorizzazione, integrità (debole), auto-sincronizzazione.
    \item Usa cifrario a flusso \textbf{RC4}.
    \item Chiave RC4 = Chiave WEP condivisa (es. 104 bit) + \textbf{IV (Initialization Vector) di 24 bit}.
    \item L'IV è inviato in chiaro.
    \item \textbf{Crittografia WEP:} (Dati + ICV) $\oplus$ Keystream. ICV è un CRC-32 (debole).
    \item \textbf{Rottura di WEP:} \textbf{Riutilizzo dell'IV}. IV a 24 bit $\rightarrow$ IV si ripetono rapidamente.
    Se $c_1 = p_1 \oplus KS_{IV}$ e $c_2 = p_2 \oplus KS_{IV}$, allora $c_1 \oplus c_2 = p_1 \oplus p_2$.
    Se $p_1$ è noto, si ricava $p_2$ e $KS_{IV}$.
\end{itemize}

\subsection{802.11i (WPA2/WPA3): Sicurezza Migliorata}
Sostituisce WEP.
\begin{itemize}
    \item Crittografia più robusta (es. AES-CCMP).
    \item Distribuzione robusta delle chiavi.
    \item Usa \textbf{Authentication Server (AS)} separato dall'Access Point (AP).
    \item \textbf{Quattro Fasi Operative di 802.11i:}
    \begin{enumerate}
        \item \textbf{Discovery:} STA e AP scoprono capacità.
        \item \textbf{Authentication:} STA e AS si autenticano (AP fa da pass-through, spesso con EAP). Generano \textbf{PMK (Pairwise Master Key)}. AS invia PMK ad AP.
        \item \textbf{Key Generation:} STA e AP usano PMK per derivare chiavi temporanee (\textbf{TK}).
        \item \textbf{Data Transfer Protetto:} Dati protetti con TK.
    \end{enumerate}
    \item \textbf{EAP (Extensible Authentication Protocol):} Protocollo end-to-end client (STA) $\leftrightarrow$ AS.
    Incapsulato: EAPoL (STA-AP), RADIUS (AP-AS).
\end{itemize}

\section{Sicurezza Operativa: Firewall e IDS}
\label{sec:firewall_ids}

\subsection{Firewall}
Isolano rete interna, permettendo/bloccando pacchetti secondo regole.

\begin{figure}[H]
    \centering
    \begin{tikzpicture}[node distance=1.5cm and 2cm, scale=0.9, transform shape]
        % Rete interna
        \node (admin_net_label) [text width=3cm, align=center] {Rete Amministrata (trusted \textquotedblleft good guys\textquotedblright)};
        \node (pc1) [computer, below left=0.5cm and -0.5cm of admin_net_label] {};
        \node (pc2) [computer, below right=0.5cm and -0.5cm of admin_net_label] {};
        \node (server_int) [server, below=0.5cm of pc1, xshift=0.7cm] {};

        % Firewall
        \node (firewall_obj) [cylinder, shape border rotate=90, draw, fill=red!70!black, minimum width=0.8cm, minimum height=1.5cm, aspect=0.3, right=2cm of server_int, yshift=0.5cm, text=white, font=\bfseries\tiny] {FW};
        \node (firewall_label) [below=0.1cm of firewall_obj, font=\small] {Firewall};

        % Internet
        \node (internet_cloud) [cloud, cloud puffs=10, draw, fill=blue!30, minimum width=3cm, minimum height=2cm, right=1.5cm of firewall_obj, text=white] {Internet};
        \node (internet_label) [below=0.1cm of internet_cloud, text width=3cm, align=center] {Internet (untrusted \textquotedblleft bad guys\textquotedblright)};

        \draw[<->, thick] ($(admin_net_label.west) - (0.5,0)$) -- (admin_net_label.east);
        \draw[<->, thick] (firewall_obj.west) -- ($(admin_net_label.east) + (1,0)$);
        \draw[<->, thick] (firewall_obj.east) -- (internet_cloud.west);
        \draw[<->, thick] (internet_cloud.west) -- ($(internet_label.east) + (0.5,0)$);

        % Connect internal devices to a conceptual switch before firewall
        \coordinate (sw_internal) at ($(server_int)!0.5!(firewall_obj) - (1cm,0)$);
        \draw (pc1.east) -- (sw_internal);
        \draw (pc2.east) -- (sw_internal);
        \draw (server_int.east) -- (sw_internal);
        \draw (sw_internal) -- (firewall_obj.west);

    \end{tikzpicture}
    \caption{Concetto di Firewall tra rete interna e Internet.}
\end{figure}


\subsubsection{Perché usare i Firewall?}
\begin{itemize}
    \item Prevenire attacchi DoS (es. SYN flooding).
    \item Prevenire modifiche/accessi illegali a dati interni.
    \item Consentire solo accessi autorizzati.
\end{itemize}

\subsubsection{Tipi di Firewall}
\begin{enumerate}
    \item \textbf{Stateless Packet Filters:}
    \begin{itemize}
        \item Router firewall, esamina pacchetti individualmente.
        \item Decisione basata su IP src/dest, porte TCP/UDP, tipo ICMP, bit TCP SYN/ACK.
        \item \textbf{ACL (Access Control Lists):} Tabelle di regole (azione, condizione).
    \end{itemize}
    \item \textbf{Stateful Packet Filters:}
    \begin{itemize}
        \item Traccia lo stato di ogni connessione TCP.
        \item Determina se i pacchetti \textquotedblleft hanno senso\textquotedblright{} nel contesto della connessione.
        \item Timeout connessioni inattive.
    \end{itemize}
    \item \textbf{Application Gateways (Proxy Firewall):}
    \begin{itemize}
        \item Filtrano sui \textbf{dati dell'applicazione} oltre che sugli header.
        \item Agiscono come proxy per specifiche applicazioni.
        \item Client deve essere configurato per usare il proxy.
    \end{itemize}
\end{enumerate}
\textbf{Esempio ACL (Stateless):}
\begin{table}[H]
    \centering
    \begin{tabular}{|l|l|l|l|l|l|l|}
        \hline
        \textbf{Azione} & \textbf{IP Src} & \textbf{IP Dest} & \textbf{Proto} & \textbf{Porta Src} & \textbf{Porta Dest} & \textbf{Flag} \\ \hline
        allow & 222.22/16 & ext. & TCP & $>$1023 & 80 & any \\ \hline
        allow & ext. & 222.22/16 & TCP & 80 & $>$1023 & ACK \\ \hline
        deny  & all & all & all & all & all & all \\ \hline
    \end{tabular}
    \caption{Esempio semplificato di Access Control List (ACL).}
\end{table}


\subsubsection{Limitazioni di Firewall e Gateway}
\begin{itemize}
    \item IP Spoofing.
    \item Complessità con molte app che necessitano gateway dedicati.
    \item Client software deve essere configurato per il proxy.
    \item Trade-off: comunicazione vs. sicurezza.
\end{itemize}

\subsection{Intrusion Detection Systems (IDS)}
Rilevano attività sospette o dannose.
\begin{itemize}
    \item \textbf{Deep Packet Inspection (DPI):} Guardano il \textbf{contenuto} dei pacchetti (vs. firme virus, stringhe attacco).
    \item \textbf{Correlazione tra pacchetti multipli:} Rilevano port scanning, network mapping, DoS.
    \item Posizionati in vari punti della rete (interna, DMZ).
\end{itemize}

\section{Riassunto sulla Sicurezza di Rete}
\label{sec:riassunto_sicurezza}

\subsection{Tecniche di Base}
\begin{itemize}
    \item \textbf{Crittografia} (simmetrica e a chiave pubblica)
    \item \textbf{Integrità dei messaggi} (hash crittografici e MAC)
    \item \textbf{Autenticazione degli endpoint}
\end{itemize}

\subsection{Usate in Diversi Scenari di Sicurezza}
\begin{itemize}
    \item Email sicure (PGP, S/MIME)
    \item Trasporto sicuro (SSL/TLS)
    \item Sicurezza a livello di rete (IPsec, VPN)
    \item Sicurezza Wireless (802.11i - WPA2/WPA3)
\end{itemize}

\subsection{Sicurezza Operativa}
\begin{itemize}
    \item Firewall
    \item Intrusion Detection Systems (IDS)
\end{itemize}

\end{document}