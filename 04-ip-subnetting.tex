%%%%%%%%%%%%%%%%%%%%%%%%%%%%%%%%%%%%%%%%%%%%%%%%%%%%%%%%%%%%%%%%%%%%%%%%%%%%%%%
% PREAMBOLO COMUNE PER APPUNTI (Stile Scuro)
%
% Questo file contiene tutte le impostazioni e i pacchetti comuni.
% NON contiene \begin{document} o \end{document}.
%
% Istruzioni per la compilazione del file principale:
% pdflatex -shell-escape nomefile_principale.tex
%%%%%%%%%%%%%%%%%%%%%%%%%%%%%%%%%%%%%%%%%%%%%%%%%%%%%%%%%%%%%%%%%%%%%%%%%%%%%%%

\documentclass{article}

% --- Encoding e lingua ---
\usepackage[utf8]{inputenc}
\usepackage[italian]{babel}

% --- Margini e layout ---
\usepackage{geometry}
\geometry{a4paper, margin=1in}

% --- Font sans-serif (come Helvetica) ---
\usepackage[scaled]{helvet}
\renewcommand{\familydefault}{\sfdefault}
\usepackage[T1]{fontenc}

% --- Matematica ---
\usepackage{amsmath}
\usepackage{amssymb}

% --- Liste personalizzate ---
\usepackage{enumitem}
% \setlist{nosep}

% --- Immagini e Grafica ---
\usepackage{float}
% \usepackage{graphicx}
\usepackage{tikz}
\usetikzlibrary{shapes.geometric, positioning, arrows.meta, calc, fit, backgrounds, patterns, decorations.pathreplacing}

% --- Tabelle Avanzate ---
\usepackage{array}
\usepackage{booktabs}
\usepackage{longtable}

% --- Hyperlink e Metadati PDF ---
\usepackage{hyperref}

\hypersetup{
    colorlinks=true,
    linkcolor=white,
    filecolor=magenta,
    urlcolor=cyan,
    citecolor=green,
    % pdftitle, pdfauthor, ecc. verranno impostati nel file principale
    pdfpagemode=FullScreen,
    bookmarksopen=true,
    bookmarksnumbered=true
}

% --- Licenza del documento ---
\usepackage[
  type={CC},
  modifier={by-sa},
  version={4.0},
]{doclicense}

% --- Colori e Sfondo Nero ---
\usepackage{xcolor}
\pagecolor{black}
\color{white}

% --- Evidenziazione del Codice ---
\usepackage{minted}
\setminted{
    frame=lines,
    framesep=2mm,
    fontsize=\small,
    breaklines=true,
    style=monokai,
    bgcolor=black!80
}
\usemintedstyle{monokai}

% --- Comandi personalizzati per algebra relazionale ---
\newcommand{\Rel}[1]{\textit{#1}} % Per i nomi delle relazioni
\newcommand{\Attr}[1]{\textsf{#1}} % Per i nomi degli attributi

\newcommand{\myunion}{\cup}
\newcommand{\myintersection}{\cap}
\newcommand{\mydifference}{-}
\newcommand{\myrename}[2]{\rho_{#1}(#2)}
\newcommand{\myselectop}[2]{\sigma_{#1}(#2)}
\newcommand{\myproject}[2]{\pi_{#1}(#2)}
\newcommand{\mycartesian}{\times}
\newcommand{\mynaturaljoin}{\bowtie} % Usare \Join da amssymb se disponibile e preferito
\newcommand{\mythetajoin}[3]{#1 \bowtie_{#2} #3} % R1 \bowtie_cond R2

% --- Comandi personalizzati per logica ---
\newcommand{\mylandop}{\wedge}
\newcommand{\myvel}{\vee}
\newcommand{\mynegop}{\neg}
\newcommand{\myforallop}{\forall}
\newcommand{\myexistsop}{\exists}

% --- Join esterni (outer join) ---
% Definizione standard per i join esterni
\def\ojoin{\setbox0=\hbox{$\mynaturaljoin$}%
	\rule[-.02ex]{.25em}{.4pt}\llap{\rule[\ht0]{.25em}{.4pt}}}
\newcommand{\myleftouterjoin}{\mathbin{\ojoin\mkern-5.8mu\mynaturaljoin}}
\newcommand{\myrightouterjoin}{\mathbin{\mynaturaljoin\mkern-5.8mu\ojoin}}
\newcommand{\myfullouterjoin}{\mathbin{\ojoin\mkern-5.8mu\mynaturaljoin\mkern-5.8mu\ojoin}}



% --- Titolo ---
\title{Indirizzamento, Subnetting e Instradamento}
\author{Basato sulle slide del Prof. Luciano Bononi}
\date{\today}

\begin{document}

\maketitle
\tableofcontents
\newpage

\section{Nello scorso episodio}

\subsection{Rappresentazione}
Un indirizzo IPv4 è un numero a 32 bit, comunemente rappresentato in \textbf{notazione decimale puntata} (es. \texttt{192.168.1.10}). Ogni numero separato da un punto rappresenta un ottetto (8 bit) e può variare da 0 a 255.

In binario, l'indirizzo \texttt{192.168.1.10} sarebbe:
\begin{minted}{text}
11000000.10101000.00000001.00001010
\end{minted}

\subsection{Classi di Indirizzi (Storiche)}
Originariamente, gli indirizzi IP erano divisi in classi:
\begin{itemize}
    \item \textbf{Classe A:} Primo bit \texttt{0}. Formato \texttt{Rete.Host.Host.Host}. Il primo ottetto (1-126) identifica la rete. Maschera di default \texttt{255.0.0.0} (o \texttt{/8}).
    \begin{itemize}
        \item Esempio: \texttt{10.x.y.z}. Rete \texttt{10.0.0.0}.
    \end{itemize}
    \item \textbf{Classe B:} Primi due bit \texttt{10}. Formato \texttt{Rete.Rete.Host.Host}. I primi due ottetti (128.0-191.255) identificano la rete. Maschera di default \texttt{255.255.0.0} (o \texttt{/16}).
    \begin{itemize}
        \item Esempio: \texttt{172.16.x.y}. Rete \texttt{172.16.0.0}.
    \end{itemize}
    \item \textbf{Classe C:} Primi tre bit \texttt{110}. Formato \texttt{Rete.Rete.Rete.Host}. I primi tre ottetti (192.0.0-223.255.255) identificano la rete. Maschera di default \texttt{255.255.255.0} (o \texttt{/24}).
    \begin{itemize}
        \item Esempio: \texttt{192.168.1.x}. Rete \texttt{192.168.1.0}.
    \end{itemize}
\end{itemize}

\section{Indirizzi IPv4: Fondamentali}

\subsection{Indirizzi Speciali all'interno di una Rete/Sottorete}
\begin{itemize}
    \item \textbf{Indirizzo di Rete (o Sottorete):} È il primo indirizzo del range, con tutti i bit della parte host a \texttt{0}. Non è assegnabile a un host.
    \begin{itemize}
        \item Esempio: \texttt{192.168.1.0} (con maschera \texttt{/24}).
    \end{itemize}
    \item \textbf{Indirizzo di Broadcast:} È l'ultimo indirizzo del range, con tutti i bit della parte host a \texttt{1}. Invia un pacchetto a tutti gli host della rete/sottorete. Non è assegnabile a un host.
    \begin{itemize}
        \item Esempio: \texttt{192.168.1.255} (con maschera \texttt{/24}).
    \end{itemize}
    \item \textbf{Indirizzo del Router (Gateway):} Per convenzione, si usa spesso il primo indirizzo host disponibile (es. \texttt{192.168.1.1}) o l'ultimo (es. \texttt{192.168.1.254}).
\end{itemize}
Per una rete con $N$ bit per la parte host, ci sono $2^N$ indirizzi totali, ma solo $2^N - 2$ sono assegnabili agli host.

\section{Maschera di Rete (Netmask)}

\subsection{Ripasso}
È una sequenza di bit \texttt{1} consecutivi seguiti da bit \texttt{0} consecutivi. Gli \texttt{1} corrispondono alla parte di rete, gli \texttt{0} alla parte host.

\subsection{Notazione CIDR (Classless Inter-Domain Routing)}
Indica il numero di bit \texttt{1} iniziali nella maschera.
\begin{itemize}
    \item \texttt{255.0.0.0} $\rightarrow$ \texttt{/8} (8 bit di rete, 24 di host)
    \item \texttt{255.255.0.0} $\rightarrow$ \texttt{/16} (16 bit di rete, 16 di host)
    \item \texttt{255.255.255.0} $\rightarrow$ \texttt{/24} (24 bit di rete, 8 di host)
    \item \texttt{255.255.255.192} $\rightarrow$ \texttt{/26} (26 bit di rete, 6 di host)
    \begin{minted}{text}
11111111.11111111.11111111.11000000
    \end{minted}
\end{itemize}

\textbf{Esempio:}

\begin{figure}[h]
    \centering
    \begin{tikzpicture}
        % IP address box
        \node[fill=blue!20, rounded corners, minimum height=0.7cm, text width=3.5cm, align=center] (ip) {\textcolor{black}{\texttt{192.168.1.0}}};
        % CIDR prefix box
        \node[fill=orange!30, rounded corners, minimum height=0.7cm, text width=1cm, right=0pt of ip, align=center] (cidr) {\textcolor{black}{\texttt{/24}}};
        
        % Legend/explanation table below
        \node[below=0.5cm of ip.south west, anchor=north west, align=left, text width=4.5cm] (table) {
            \begin{tabular}{|l|l|}
                \hline
                \cellcolor{blue!20} \textcolor{black}{\textbf{192.168.1.0}} & Indirizzo IP \\
                \hline
                \cellcolor{orange!30} \textcolor{black}{\textbf{/24}} & Prefisso di rete \\
                \hline
            \end{tabular}
        };
        
        % Draw an arrow pointing to the notation
        \draw[->, thick] (ip.north) -- ++(0,0.5) -| (2.5,1.2) node[above] {Notazione CIDR};
    \end{tikzpicture}
    \caption{Esempio di notazione CIDR}
\end{figure}

\subsection{Validità delle Maschere}
Una maschera è valida se è composta da una sequenza ininterrotta di \texttt{1} seguita da una sequenza ininterrotta di \texttt{0}.
\begin{itemize}
    \item \texttt{255.255.255.0} (\texttt{...11111111.00000000}) $\rightarrow$ \textbf{VALIDA}
    \item \texttt{255.255.128.0} (\texttt{...10000000.00000000}) $\rightarrow$ \textbf{VALIDA}
    \item \texttt{255.255.128.128} (\texttt{...10000000.10000000}) $\rightarrow$ \textbf{NON VALIDA}
\end{itemize}

\subsection{Ottenere l'Indirizzo di Rete/Sottorete}
Si ottiene eseguendo un'operazione di \textbf{AND logico} bit a bit tra l'indirizzo IP dell'host e la sua maschera di rete.

\textbf{Esempio:} Host \texttt{192.168.1.77}, Maschera \texttt{255.255.255.0}
\begin{itemize}
    \item IP: \texttt{11000000.10101000.00000001.01001101}
    \item Mask: \texttt{11111111.11111111.11111111.00000000}
    \item Risultato AND (Indirizzo di Sottorete):
    \begin{minted}{text}
11000000.10101000.00000001.00000000  ->  192.168.1.0
    \end{minted}
\end{itemize}

\section{Subnetting: Creazione di Sottoreti}

\subsection{Scopo}
Il subnetting permette di dividere una rete IP più grande in sottoreti più piccole per:
\begin{itemize}
    \item Organizzare meglio la rete.
    \item Ridurre il traffico di broadcast.
    \item Migliorare la sicurezza.
\end{itemize}

\subsection{Come Funziona}
Si "rubano" bit dalla parte host dell'indirizzo per usarli come bit di sottorete. Questo aumenta la parte di rete della maschera.
\begin{itemize}
    \item Se si rubano $k$ bit, si possono creare $2^k$ sottoreti.
    \item Se la parte host originale aveva $H$ bit, ora avrà $H-k$ bit. Ogni sottorete potrà ospitare $2^{(H-k)} - 2$ host.
\end{itemize}

\subsection{Esempio Pratico: Dividere \texttt{130.136.0.0/16} in 8 sottoreti}
\begin{enumerate}
    \item \textbf{Bit necessari per le sottoreti:} Per 8 sottoreti, servono $\log_2(8) = 3$ bit.
    \item \textbf{Nuova maschera:} La maschera originale \texttt{/16}. Rubiamo 3 bit dalla parte host. La nuova maschera avrà $16 + 3 = 19$ bit (\texttt{/19}).
    \begin{itemize}
        \item Originale (\texttt{/16}): \texttt{11111111.11111111.00000000.00000000}
        \item Nuova (\texttt{/19}): \texttt{11111111.11111111.11100000.00000000} $\rightarrow$ \texttt{255.255.224.0}
    \end{itemize}
    \item \textbf{Host per sottorete:} Bit per host rimanenti: $32 - 19 = 13$.
    Ogni sottorete può avere $2^{13} - 2 = 8192 - 2 = 8190$ host.
    \item \textbf{Range delle sottoreti:} (identificate dai 3 bit rubati \texttt{sss} nel terzo ottetto: \texttt{130.136.sssxxxxx.xxxxxxxx})
    \begin{itemize}
        \item \texttt{000}: \texttt{130.136.0.0/19} (range host \texttt{130.136.0.1} - \texttt{130.136.31.254})
        \item \texttt{001}: \texttt{130.136.32.0/19} (range host \texttt{130.136.32.1} - \texttt{130.136.63.254})
        \item \dots
        \item \texttt{111}: \texttt{130.136.224.0/19} (range host \texttt{130.136.224.1} - \texttt{130.136.255.254})
    \end{itemize}
\end{enumerate}
\textbf{Esempio di assegnazione:} L'host \texttt{130.136.169.4} con maschera \texttt{255.255.224.0} (\texttt{/19}):
\begin{itemize}
    \item IP: \texttt{10000010.10001000.10101001.00000100} (terzo ottetto: \texttt{10101001} = 169)
    \item Maschera: \texttt{11111111.11111111.11100000.00000000}
    \item AND (per sottorete): \texttt{10000010.10001000.10100000.00000000} $\rightarrow$ \texttt{130.136.160.0}.
    I bit di sottorete nel terzo ottetto sono \texttt{101}.
\end{itemize}


\subsection{Esercizio: Progettare una Sottorete per 17 host}
Data la rete di Classe C \texttt{212.11.24.0/24}.
\begin{enumerate}
    \item \textbf{Host necessari:} Per 17 host, il "contenitore" (potenza di 2) più piccolo è $2^5 = 32$. Questo lascia 5 bit per la parte host (32 indirizzi totali, 30 usabili).
    \item \textbf{Bit da rubare:} La rete \texttt{/24} ha 8 bit per gli host. Se ne usiamo 5 per gli host, ne "rubiamo" $8 - 5 = 3$ per le sottoreti.
    \item \textbf{Nuova maschera:} Maschera originale \texttt{/24}. Nuova maschera \texttt{/24} + 3 = \texttt{/27}.
    \begin{itemize}
        \item \texttt{/27} $\rightarrow$ \texttt{11111111.11111111.11111111.11100000} $\rightarrow$ \texttt{255.255.255.224}.
    \end{itemize}
    \item \textbf{Numero di sottoreti create:} $2^3 = 8$ sottoreti.
    \item \textbf{Allocazione (esempio):}
    \begin{itemize}
        \item Sottorete 0: \texttt{212.11.24.0/27} (host \texttt{212.11.24.1} - \texttt{212.11.24.30})
        \item Sottorete 1: \texttt{212.11.24.32/27} (host \texttt{212.11.24.33} - \texttt{212.11.24.62})
        \item \dots e così via.
    \end{itemize}
\end{enumerate}

\subsection{Supernetting (Superrete)}
Il supernetting, o superrete, è il processo inverso al subnetting. Consiste nel combinare più reti più piccole e contigue in un unico blocco di indirizzi più grande, rappresentato da una singola route (route aggregation/summarization).
\begin{itemize}
    \item \textbf{Scopo:} Ridurre la dimensione delle tabelle di routing nei router. Se un router ha più rotte verso reti contigue che possono essere aggregate, può sostituirle con una singola route "supernet".
    \item \textbf{Come Funziona:} Si "restituiscono" bit dalla parte di rete alla parte host, diminuendo la lunghezza del prefisso della maschera (es. da \texttt{/24} a \texttt{/22}).
    \item \textbf{Esempio:} Le reti \texttt{192.168.0.0/24} e \texttt{192.168.1.0/24} possono essere aggregate in \texttt{192.168.0.0/23}.
    Per farlo, si identificano i bit comuni nella parte di rete.
    \begin{minted}{text}
192.168.0.0 = 11000000.10101000.00000000.00000000
192.168.1.0 = 11000000.10101000.00000001.00000000
    \end{minted}
    I primi 23 bit sono uguali. Quindi, la super-rete è \texttt{192.168.0.0/23}, che copre gli indirizzi da \texttt{192.168.0.0} a \texttt{192.168.1.255}.
    La nuova maschera \texttt{/23} è \texttt{255.255.254.0}.
\end{itemize}

\section{Routing (Instradamento dei Pacchetti)}

\subsection{Decisione Iniziale del Mittente}
Quando un host (mittente) deve inviare un pacchetto:
\begin{enumerate}
    \item \textbf{Confronto:} Il mittente calcola l'indirizzo di rete del destinatario usando la \textbf{PROPRIA} maschera di rete.
    \begin{itemize}
        \item \texttt{Rete\_Mittente = IP\_Mittente AND Maschera\_Mittente}
        \item \texttt{Rete\_Destinatario\_Vista\_Dal\_Mittente = IP\_Destinatario AND Maschera\_Mittente}
    \end{itemize}
    \item \textbf{Decisione:}
    \begin{itemize}
        \item Se \texttt{Rete\_Mittente == Rete\_Destinatario\_Vista\_Dal\_Mittente}: il destinatario è sulla \textbf{stessa rete locale/sottorete}. Il pacchetto viene inviato direttamente (usando ARP).
        \item Se \texttt{Rete\_Mittente != Rete\_Destinatario\_Vista\_Dal\_Mittente}: il destinatario è su una \textbf{rete esterna}. Il pacchetto viene inviato al \textbf{Default Gateway} (router di default).
    \end{itemize}
\end{enumerate}
\textbf{Esempio:} Mittente \texttt{140.217.2.10} (mask \texttt{255.255.255.0}), Destinatario \texttt{130.136.2.33}.
\begin{itemize}
    \item \texttt{Rete\_Mittente = 140.217.2.10 AND 255.255.255.0 = 140.217.2.0}
    \item \texttt{Rete\_Dest\_Vista = 130.136.2.33 AND 255.255.255.0 = 130.136.2.0}
    \item \texttt{140.217.2.0 != 130.136.2.0} $\rightarrow$ Rete esterna. Invia al gateway (es. \texttt{140.217.2.254}).
\end{itemize}

\subsection{Processo Decisionale del Router}
Quando un router riceve un pacchetto, segue questo processo decisionale:

\begin{table}[h]
    \centering
    \begin{tabular}{|p{2.5cm}|p{5cm}|p{6cm}|}
    \hline
    \rowcolor{bg_custom}
    \textbf{Fase} & \textbf{Operazione} & \textbf{Dettaglio} \\
    \hline
    1. Analisi & \textbf{Controllo Destinazione} & Il router esamina l'indirizzo IP di destinazione nel pacchetto \\
    \hline
    2. Verifica & \textbf{Confronto con Interfacce} & Per ogni interfaccia calcola:\\
    \texttt{IP\_Dest AND Maschera\_Interfaccia} & \\
    \hline
    \multirow{4}{*}{3. Decisione} & \multirow{4}{*}{\textbf{Scelta Percorso}} & \textbf{a)} Se la destinazione è su una rete direttamente connessa: inoltra sulla relativa interfaccia \\
    \cline{3-3}
    & & \textbf{b)} Altrimenti: consulta la tabella di routing \\
    \cline{3-3}
    & & \textbf{c)} Cerca corrispondenza più specifica (\textit{longest prefix match}) \\
    \cline{3-3}
    & & \textbf{d)} Se nessuna corrispondenza: usa route di default o scarta \\
    \hline
    \end{tabular}
    \caption{Processo decisionale del router per l'instradamento}
\end{table}

\begin{figure}[h]
    \centering
    \begin{tikzpicture}[node distance=1.5cm, 
                        block/.style={rectangle, draw, text width=5cm, text centered, rounded corners, minimum height=1cm, fill=blue!10},
                        decision/.style={diamond, draw, text width=4.5cm, text badly centered, inner sep=0pt, fill=orange!20},
                        arrow/.style={thick,->,>=stealth}]
                        
        % Nodi
        \node[block] (start) {\textcolor{black}{Router riceve un pacchetto}};
        \node[block, below of=start] (check) {\textcolor{black}{Controlla IP di destinazione}};
        \node[decision, below=1cm of check, node distance=2.5cm] (connected) {\textcolor{black}{È collegato a una rete direttamente connessa?}};
        \node[block, right=2cm of connected] (direct) {\textcolor{black}{Invia sulla relativa interfaccia}};
        \node[block, below=1cm of connected] (routing) {\textcolor{black}{Consulta routing table}};
        \node[decision, below=1cm of routing] (route) {\textcolor{black}{Esiste una route conosciuta?}};
        \node[block, right=2cm of route] (specific) {\textcolor{black}{Usa la route più specifica (longest prefix match)}};
        \node[decision, below=1cm of route] (default) {\textcolor{black}{Esiste route di default?}};
        \node[block, right=2cm of default] (usedefault) {\textcolor{black}{Usa default route}};
        \node[block, below=1cm of default] (discard) {\textcolor{black}{Scarta pacchetto (ICMP Destination Unreachable)}};
        
        % Collegamenti
        \draw[arrow] (start) -- (check);
        \draw[arrow] (check) -- (connected);
        \draw[arrow] (connected) -- node[anchor=south] {Sì} (direct);
        \draw[arrow] (connected) -- node[anchor=east] {No} (routing);
        \draw[arrow] (routing) -- (route);
        \draw[arrow] (route) -- node[anchor=south] {Sì} (specific);
        \draw[arrow] (route) -- node[anchor=east] {No} (default);
        \draw[arrow] (default) -- node[anchor=south] {Sì} (usedefault);
        \draw[arrow] (default) -- node[anchor=east] {No} (discard);
    \end{tikzpicture}
    \caption{Flowchart del processo decisionale del router}
\end{figure}
  

\begin{figure}[h]
    \centering
    \begin{tikzpicture}[
        transform shape,
        router/.style={rectangle, draw, thick, fill=blue!20, minimum width=1.7cm, minimum height=1cm, rounded corners},
        network/.style={cloud, draw, thick, fill=green!10, minimum width=2cm, minimum height=1.2cm, cloud puffs=10, text=black},
        host/.style={circle, draw, thick, fill=yellow!20, minimum size=0.7cm},
        arrow/.style={->, thick, >=stealth},
        every node/.style={text=black},
        packet/.style={rectangle, draw, thick, fill=red!10, minimum width=0.5cm, minimum height=0.3cm, font=\scriptsize}
    ]
    
    % Networks - Adjusted positions
    \node[network, align=center] (net1) at (-2,4) {Rete A\\192.168.1.0/24};
    \node[network, align=center] (net2) at (8,4) {Rete B\\10.0.0.0/24};
    \node[network, align=center] (net3) at (3,0) {Rete C\\172.16.0.0/24};
    
    % Routers - Adjusted positions
    \node[router, align=center] (r1) at (1,2) {Router R1\\192.168.1.1\\10.0.0.254};
    \node[router, align=center] (r2) at (5,2) {Router R2\\10.0.0.1\\172.16.0.254};
    
    % Hosts - Adjusted positions
    \node[host] (h1) at (-2,2.5) {H1};
    \node[host] (h2) at (8,2.5) {H2};
    \node[host] (h3) at (3,-1.5) {H3};
    
    % IP addresses - Adjusted positions
    \node[font=\footnotesize] at (-2,2) {192.168.1.10};
    \node[font=\footnotesize] at (8,2) {10.0.0.20};
    \node[font=\footnotesize] at (3,-2) {172.16.0.30};
    
    % Connections - Made more direct
    \draw[arrow] (net1) -- (r1);
    \draw[arrow] (r1) -- (net2);
    \draw[arrow] (net2) -- (r2);
    \draw[arrow] (r2) -- (net3);
    \draw[arrow] (h1) -- (net1);
    \draw[arrow] (h2) -- (net2);
    \draw[arrow] (h3) -- (net3);
    
    % Decision process - Moved higher and adjusted width
    \node[align=left, fill=white, draw, rounded corners] at (3,7) {
        1. H1 invia pacchetto per H3 (172.16.0.30)\\
        2. H1 confronta rete: $192.168.1.10 \land 255.255.255.0 \neq 172.16.0.30 \land 255.255.255.0$\\
        3. H1 invia al gateway (R1)\\
        4. R1 consulta tabella, inoltra a R2\\
        5. R2 rileva rete direttamente connessa\\
        6. R2 consegna a H3
    };
    
    \end{tikzpicture}
    \caption{Esempio di instradamento attraverso più reti}
\end{figure}

\textbf{Esempio di instradamento (dalle slide):} Pacchetto da \texttt{140.217.2.10} a \texttt{130.136.2.33}.
\begin{itemize}
    \item \textbf{Host Mittente (\texttt{140.217.2.10}, Mask \texttt{255.255.255.0}):}
    \begin{itemize}
        \item Rete mittente: \texttt{140.217.2.0}
        \item Destinazione (\texttt{130.136.2.33}) AND Mask mittente (\texttt{255.255.255.0}) = \texttt{130.136.2.0}
        \item \texttt{140.217.2.0 != 130.136.2.0} $\rightarrow$ Destinazione esterna. Invia a Default Gateway \texttt{Ry2} (\texttt{140.217.2.254}).
    \end{itemize}
    \item \textbf{Router Ry2 (Default Gateway del mittente, IP interf. \texttt{140.217.2.254}, mask \texttt{255.255.255.0}):}
    \begin{itemize}
        \item Destinazione (\texttt{130.136.2.33}) AND Mask interfaccia (\texttt{255.255.255.0}) = \texttt{130.136.2.0}.
        \item Rete connessa a Ry2 è \texttt{140.217.2.0}. Sono diverse.
        \item Ry2 ha un default router, Ry (\texttt{140.217.0.254}). Inoltra a Ry.
    \end{itemize}
    \item \textbf{Router Ry (Default Router di Ry2, IP interf. es. \texttt{140.217.0.254}, mask interf. \texttt{255.255.0.0}):}
    \begin{itemize}
        \item Destinazione (\texttt{130.136.2.33}) AND Mask interfaccia (\texttt{255.255.0.0}) = \texttt{130.136.0.0}.
        \item Rete di Ry è \texttt{140.217.0.0}. Sono diverse.
        \item Ry consulta la sua tabella: trova route per \texttt{130.136.0.0/16} via router Rz (\texttt{190.89.0.254}). Inoltra a Rz.
    \end{itemize}
    \item \textbf{Router Rz (IP interf. es. \texttt{190.89.0.254}, mask interf. \texttt{255.255.0.0}):}
    \begin{itemize}
        \item Destinazione (\texttt{130.136.2.33}) AND Mask interfaccia (\texttt{255.255.0.0}) = \texttt{130.136.0.0}.
        \item Rete di Rz è \texttt{190.89.0.0}. Sono diverse.
        \item Rz consulta la tabella: trova route per \texttt{130.136.0.0/16} via router Rk (\texttt{130.136.0.254}). Inoltra a Rk.
    \end{itemize}
    \item \textbf{Router Rk (IP interf. \texttt{130.136.0.254}, mask interf. \texttt{255.255.0.0}):}
    \begin{itemize}
        \item Destinazione (\texttt{130.136.2.33}) AND Mask interfaccia (\texttt{255.255.0.0}) = \texttt{130.136.0.0}.
        \item Rete di Rk è \texttt{130.136.0.0}. Corrispondono!
        \item Rk sa che la sottorete \texttt{130.136.2.0/24} è gestita dal router Rk2 (\texttt{130.136.2.254}). Inoltra a Rk2.
    \end{itemize}
    \item \textbf{Router Rk2 (IP interf. \texttt{130.136.2.254}, mask interf. \texttt{255.255.255.0}):}
    \begin{itemize}
        \item Destinazione (\texttt{130.136.2.33}) AND Mask interfaccia (\texttt{255.255.255.0}) = \texttt{130.136.2.0}.
        \item Rete di Rk2 è \texttt{130.136.2.0}. Corrispondono!
        \item Rk2 usa ARP per trovare il MAC di \texttt{130.136.2.33} e consegna il pacchetto.
    \end{itemize}
\end{itemize}

\subsection{ARP (Address Resolution Protocol)}
Usato per trovare l'indirizzo MAC (fisico, di livello 2) di un host o router conoscendo il suo indirizzo IP, \textbf{solo all'interno della stessa rete locale/sottorete}.
\begin{itemize}
    \item Host A deve inviare a IP B (stessa rete), ma non conosce MAC di B.
    \item A invia richiesta ARP: "Chi ha l'IP B?" (broadcast MAC).
    \item Host B risponde: "Io ho l'IP B, il mio MAC è M\_B" (unicast MAC ad A).
    \item A incapsula il pacchetto IP in un frame Ethernet con MAC destinazione M\_B.
\end{itemize}

\section{Conversioni di Base Numerica e Operazioni Binarie}

\subsection{\texorpdfstring{Decimale $\rightarrow$ Binario}{Decimale -> Binario}}
Metodo delle divisioni successive per 2, leggendo i resti dal basso verso l'alto.
\textbf{Esempio:} $14917_{10}$
\begin{itemize}
    \item $14917 / 2 = 7458$ resto $1$
    \item $7458 / 2 = 3729$ resto $0$
    \item $3729 / 2 = 1864$ resto $1$
    \item \dots (e così via)
\end{itemize}
$14917_{10} = 11101001000101_2$. (Servono 14 bit: $\lceil \log_2(14917) \rceil = 14$).

\subsection{\texorpdfstring{Binario $\rightarrow$ Decimale}{Binario -> Decimale}}
Sommare le potenze di 2 corrispondenti ai bit \texttt{1}.
\textbf{Esempio:} $1011_2 = 1 \cdot 2^3 + 0 \cdot 2^2 + 1 \cdot 2^1 + 1 \cdot 2^0 = 8 + 0 + 2 + 1 = 11_{10}$.

\subsection{\texorpdfstring{Binario $\rightarrow$ Ottale (base 8)}{Binario -> Ottale (base 8)}}
Raggruppare i bit binari in gruppi di 3 da destra.
\textbf{Esempio:} $101001010101_2$
\begin{itemize}
    \item \texttt{101 001 010 101}
    \item \texttt{ 5   1   2   5 }
\end{itemize}
Risultato: $5125_8$.

\subsection{\texorpdfstring{Binario $\rightarrow$ Esadecimale (base 16)}{Binario -> Esadecimale (base 16)}}
Raggruppare i bit binari in gruppi di 4 da destra.
\textbf{Esempio:} $101001010101_2$
\begin{itemize}
    \item \texttt{(00)10 1001 0101} (Aggiunti due 0 a sinistra)
    \item \texttt{   2    9    5  }
\end{itemize}
Risultato: $295_{16}$.

\subsection{Operazione Logica AND}
\begin{itemize}
    \item \texttt{0 AND 0 = 0}
    \item \texttt{0 AND 1 = 0}
    \item \texttt{1 AND 0 = 0}
    \item \texttt{1 AND 1 = 1}
\end{itemize}

\end{document}