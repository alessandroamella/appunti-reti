\documentclass{article}

% --- Encoding e lingua ---
\usepackage[utf8]{inputenc}
\usepackage[italian]{babel}

% --- Margini e layout ---
\usepackage{geometry}
\geometry{a4paper, margin=1in}

% --- Font sans-serif ---
\usepackage[scaled]{helvet}
\renewcommand{\familydefault}{\sfdefault}
\usepackage[T1]{fontenc}

% --- Matematica ---
\usepackage{amsmath}
\usepackage{amssymb}

% --- Liste personalizzate ---
\usepackage{enumitem}

% --- Immagini ---
\usepackage{float}
\usepackage{tikz} % per disegni
\usepackage{graphicx} % per immagini
\usetikzlibrary{shapes.geometric, positioning, calc}

% --- Hyperlink ---
\usepackage{hyperref}
\hypersetup{
    pdftitle={Appunti su Progettazione Reti IPv4 e Subnetting},
    colorlinks=true,
    linkcolor=cyan,
    filecolor=magenta,      
    urlcolor=green,
    citecolor=blue,
}

% --- Colori e sfondo nero ---
\usepackage{xcolor}
\pagecolor{black}
\color{white}

% --- Evidenziazione del codice (richiede -shell-escape) ---
% Compilare con: pdflatex -shell-escape nomefile.tex
\usepackage{minted}
\setminted{
    frame=lines,     % Cornice attorno al codice
    framesep=2mm,
    fontsize=\small,
    breaklines=true, % A capo automatico per linee lunghe
    style=monokai    % Stile di highlighting
}

% --- Titolo ---
\title{Appunti su Progettazione Reti IPv4 e Subnetting}
\author{Basato sulle slide del Prof. Luciano Bononi}
\date{\today}

\begin{document}

\maketitle
\tableofcontents
\newpage

\section{Introduzione alla Progettazione di Reti IPv4 e Subnetting (VLSM)}
L'obiettivo della progettazione di reti IPv4 è allocare in modo efficiente lo spazio di indirizzamento IP disponibile per soddisfare i requisiti di host di diverse sottoreti (subnet), minimizzando lo spreco di indirizzi. Questo approccio utilizza spesso il Variable Length Subnet Masking (VLSM).

\subsection{Concetti Fondamentali}
\begin{itemize}
    \item \textbf{Indirizzo di Rete (Network Address):} Il primo indirizzo di un blocco IP, con tutti i bit della parte host a 0. Non è assegnabile a un host. Esempio: \texttt{192.168.1.0}.
    \item \textbf{Indirizzo di Broadcast:} L'ultimo indirizzo di un blocco IP, con tutti i bit della parte host a 1. Invia dati a tutti gli host in quella specifica sottorete. Non è assegnabile a un host. Esempio: \texttt{192.168.1.255} per una rete /24.
    \item \textbf{Host Utilizzabili:} Gli indirizzi compresi tra l'indirizzo di rete e l'indirizzo di broadcast. Il numero di host utilizzabili è $(2^h) - 2$, dove $h$ è il numero di bit dedicati alla parte host.
    \item \textbf{Netmask (Maschera di Sottorete):} Una sequenza di bit che distingue la parte di rete dalla parte host in un indirizzo IP. I bit a 1 indicano la parte di rete, i bit a 0 la parte host. Esempio: \texttt{255.255.255.0}.
    \item \textbf{Notazione CIDR (Classless Inter-Domain Routing):} Indica il numero di bit a 1 consecutivi nella netmask (cioè i bit della parte di rete). Esempio: \texttt{/24} corrisponde a \texttt{255.255.255.0}.
    \item \textbf{Router:} Un dispositivo che connette diverse reti o sottoreti. Ogni interfaccia del router ha un indirizzo IP che appartiene alla sottorete a cui è connessa e funge da gateway per gli host di quella sottorete. Solitamente si assegna il primo o l'ultimo indirizzo IP \textit{utilizzabile} della sottorete al router.
    \item \textbf{Router di Default (Default Gateway):} L'indirizzo IP del router che gli host di una sottorete usano per comunicare con host esterni alla loro sottorete.
\end{itemize}

\subsection{Processo di Subnetting con VLSM}
VLSM permette di usare maschere di sottorete di lunghezza variabile per diverse sottoreti, ottimizzando l'uso degli indirizzi. La strategia generale è:
\begin{enumerate}
    \item \textbf{Elencare i Requisiti:} Determinare il numero di host necessari per ogni sottorete.
    \item \textbf{Ordinare per Dimensione:} Ordinare le sottoreti dalla più grande (più host richiesti) alla più piccola. Questo aiuta a prevenire la frammentazione dello spazio di indirizzamento.
    \item \textbf{Calcolare la Dimensione del Blocco:} Per ogni sottorete, calcolare il numero minimo di indirizzi totali necessari: $N_{\text{host richiesti}} + 2$ (per indirizzo di rete e broadcast). Trovare la più piccola potenza di 2 ($2^h$) che sia maggiore o uguale a questo valore. $h$ sarà il numero di bit per la parte host.
    \item \textbf{Determinare la Netmask/Prefisso:} Se $h$ sono i bit di host, allora $32 - h$ sono i bit di rete (il prefisso CIDR). Da qui si deriva la netmask.
    \item \textbf{Allocare gli Indirizzi:} Iniziare ad allocare blocchi di indirizzi contigui dal range disponibile, partendo dalla sottorete più grande.
\end{enumerate}

\newpage
\section{Esempio 1: Rete N: \texttt{199.201.17.0 / 24}}
Si dispone di un blocco iniziale \texttt{199.201.17.0 / 24}.
\begin{minted}{text}
Netmask: 255.255.255.0
Indirizzi totali: 256 (da .0 a .255)
Host utilizzabili: 254 (da .1 a .254)
Indirizzo di Rete N: 199.201.17.0
Primo Host N: 199.201.17.1
Ultimo Host N (convenzionale): 199.201.17.253
Router N (convenzionale): 199.201.17.254
Broadcast N: 199.201.17.255
\end{minted}

\subsection{Suddivisione}
La suddivisione considera la gerarchia delle reti e alloca blocchi contigui.

\subsubsection{Sottorete N2A (42 hosts)}
\begin{itemize}
    \item Host richiesti: 42. Indirizzi totali: $42 + 2 = 44$.
    \item Potenza di 2 $\geq 44$: $64 = 2^6$. Quindi $h=6$ bit per host.
    \item Prefisso: $32 - 6 = /26$. Netmask: \texttt{255.255.255.192}.
    \item Allocazione N2A: \texttt{199.201.17.0 / 26}.
\end{itemize}
\begin{minted}{text}
Rete N2A:
Network:    199.201.17.0
Netmask:    255.255.255.192
First Host: 199.201.17.1
Last Host:  199.201.17.61
Router:     199.201.17.62
Broadcast:  199.201.17.63
\end{minted}

\subsubsection{Sottorete N2 (119 hosts, include N2A)}
\begin{itemize}
    \item Host richiesti (totali in N2): 119. Indirizzi totali: $119 + 2 = 121$.
    \item Potenza di 2 $\geq 121$: $128 = 2^7$. Quindi $h=7$ bit per host.
    \item Prefisso: $32 - 7 = /25$. Netmask: \texttt{255.255.255.128}.
    \item N2 ingloba N2A (\texttt{.0} - \texttt{.63}). Allocazione N2: \texttt{199.201.17.0 / 25}.
\end{itemize}
\begin{minted}{text}
Rete N2:
Network:    199.201.17.0
Netmask:    255.255.255.128
First Host: 199.201.17.1 (in N2A)
Last Host:  199.201.17.125
Router:     199.201.17.126 (per host in N2 ma non N2A)
Broadcast:  199.201.17.127
\end{minted}

\subsubsection{Sottorete N1B (15 hosts)}
Il prossimo blocco disponibile dopo N2 (\texttt{.0} - \texttt{.127}) inizia da \texttt{199.201.17.128}.
\begin{itemize}
    \item Host richiesti: 15. Indirizzi totali: $15 + 2 = 17$.
    \item Potenza di 2 $\geq 17$: $32 = 2^5$. Quindi $h=5$ bit per host.
    \item Prefisso: $32 - 5 = /27$. Netmask: \texttt{255.255.255.224}.
    \item Allocazione N1B: \texttt{199.201.17.128 / 27}.
\end{itemize}
\begin{minted}{text}
Rete N1B:
Network:    199.201.17.128
Netmask:    255.255.255.224
First Host: 199.201.17.129
Last Host:  199.201.17.157
Router:     199.201.17.158
Broadcast:  199.201.17.159
\end{minted}

\subsubsection{Sottorete N1A (6 hosts)}
Il prossimo blocco disponibile dopo N1B (\texttt{.128} - \texttt{.159}) inizia da \texttt{199.201.17.160}.
\begin{itemize}
    \item Host richiesti: 6. Indirizzi totali: $6 + 2 = 8$.
    \item Potenza di 2 $\geq 8$: $8 = 2^3$. Quindi $h=3$ bit per host.
    \item Prefisso: $32 - 3 = /29$. Netmask: \texttt{255.255.255.248}.
    \item Allocazione N1A: \texttt{199.201.17.160 / 29}.
\end{itemize}
\begin{minted}{text}
Rete N1A:
Network:    199.201.17.160
Netmask:    255.255.255.248
First Host: 199.201.17.161
Last Host:  199.201.17.165
Router:     199.201.17.166
Broadcast:  199.201.17.167
\end{minted}

\subsubsection{Sottorete N1 (64 hosts, include N1A e N1B)}
\begin{itemize}
    \item Host richiesti (totali in N1): 64. Indirizzi totali: $64 + 2 = 66$.
    \item Potenza di 2 $\geq 66$: $128 = 2^7$. Quindi $h=7$ bit per host.
    \item Prefisso: $32 - 7 = /25$. Netmask: \texttt{255.255.255.128}.
    \item N1 ingloba N1B (\texttt{.128} - \texttt{.159}) e N1A (\texttt{.160} - \texttt{.167}). Allocazione N1: \texttt{199.201.17.128 / 25}.
\end{itemize}
\begin{minted}{text}
Rete N1:
Network:    199.201.17.128
Netmask:    255.255.255.128
First Host: 199.201.17.129 (in N1B)
Last Host:  199.201.17.252 (per host in N1 ma non N1A/N1B)
Router:     199.201.17.253 (per host in N1 ma non N1A/N1B)
Broadcast:  199.201.17.255
\end{minted}

\subsection{Considerazioni sull'Esempio 1}
\begin{itemize}
    \item \textbf{Conflitto di Broadcast?:} La rete N1 (\texttt{199.201.17.128/25}) ha come broadcast \texttt{199.201.17.255}. Anche la rete padre N (\texttt{199.201.17.0/24}) ha come broadcast \texttt{199.201.17.255}. Questo non è problematico perché ogni host interpreta l'indirizzo di broadcast in base alla propria netmask. Dato che N1 e N2 coprono l'intero spazio /24, non ci sono host "solo in N".
    \item \textbf{Router di Default:}
    \begin{itemize}
        \item Host in N1A usano \texttt{199.201.17.166} come gateway.
        \item Host in N1B usano \texttt{199.201.17.158} come gateway.
        \item Host in N1 (ma non N1A/N1B) usano \texttt{199.201.17.253} come gateway.
        \item I router \texttt{.166} e \texttt{.158} avranno come default gateway \texttt{199.201.17.253}.
        \item Il router di N1 (\texttt{.253}) e il router di N2 (\texttt{.126}) avranno come default gateway il router principale della rete N (\texttt{199.201.17.254} o un router del provider).
    \end{itemize}
\end{itemize}

\newpage
\section{Esempio 2: Rete \texttt{130.136.0.0 / 16}}
Si dispone di un blocco iniziale \texttt{130.136.0.0 / 16}.
\begin{minted}{text}
Netmask: 255.255.0.0
Indirizzi totali: 2^16 = 65536
\end{minted}

\subsection{Requisiti}
\begin{itemize}
    \item LAN IP2: 260 host (divisa in IP2-B: 140 host, IP2-A: 120 host)
    \item LAN IP1: 48 host
    \item LAN IP3: 4 host
\end{itemize}

\subsection{Ordine di allocazione}
Si alloca partendo dalla sottorete con più host richiesti, e si procede con blocchi contigui.

\subsubsection{Sottorete IP2-B (140 hosts)}
\begin{itemize}
    \item Host richiesti: 140. Indirizzi totali: $140 + 2 = 142$.
    \item Potenza di 2 $\geq 142$: $256 = 2^8$. $h=8$ bit per host.
    \item Prefisso: $/24$. Netmask: \texttt{255.255.255.0}.
    \item Allocazione IP2-B: \texttt{130.136.0.0 / 24}.
\end{itemize}
\begin{minted}{text}
Rete IP2-B:
Network:    130.136.0.0
Netmask:    255.255.255.0
First Host: 130.136.0.1
Last Host:  130.136.0.253
Router:     130.136.0.254
Broadcast:  130.136.0.255
\end{minted}

\subsubsection{Sottorete IP2-A (120 hosts)}
Prossimo blocco dopo IP2-B (che finisce a \texttt{130.136.0.255}).
\begin{itemize}
    \item Host richiesti: 120. Indirizzi totali: $120 + 2 = 122$.
    \item Potenza di 2 $\geq 122$: $128 = 2^7$. $h=7$ bit per host.
    \item Prefisso: $/25$. Netmask: \texttt{255.255.255.128}.
    \item Allocazione IP2-A: \texttt{130.136.1.0 / 25}.
\end{itemize}
\begin{minted}{text}
Rete IP2-A:
Network:    130.136.1.0
Netmask:    255.255.255.128
First Host: 130.136.1.1
Last Host:  130.136.1.125
Router:     130.136.1.126
Broadcast:  130.136.1.127
\end{minted}

\subsubsection{Sottorete IP2 (260 hosts, include IP2-A e IP2-B)}
IP2 deve contenere IP2-B (\texttt{130.136.0.0/24}) e IP2-A (\texttt{130.136.1.0/25}).
\begin{itemize}
    \item Range coperto: da \texttt{130.136.0.0} a \texttt{130.136.1.127}. Totale indirizzi usati da IP2-A/B = $256 + 128 = 384$.
    \item Potenza di 2 $\geq 384$: $512 = 2^9$. $h=9$ bit per host.
    \item Prefisso: $/23$. Netmask: \texttt{255.255.254.0}.
    \item Allocazione IP2: \texttt{130.136.0.0 / 23} (copre \texttt{130.136.0.0} - \texttt{130.136.1.255}).
\end{itemize}
\begin{minted}{text}
Rete IP2:
Network:    130.136.0.0
Netmask:    255.255.254.0
First Host: 130.136.0.1 (in IP2-B)
Last Host:  130.136.1.253 (per host in IP2 ma non IP2-A/B)
Router:     130.136.1.254 (per host in IP2 ma non IP2-A/B)
Broadcast:  130.136.1.255
Host in IP2 ma non in IP2-A/B: da 130.136.1.128 a 130.136.1.253
\end{minted}

\subsubsection{Sottorete IP1 (48 hosts)}
Prossimo blocco dopo IP2 (che finisce a \texttt{130.136.1.255}).
\begin{itemize}
    \item Host richiesti: 48. Indirizzi totali: $48 + 2 = 50$.
    \item Potenza di 2 $\geq 50$: $64 = 2^6$. $h=6$ bit per host.
    \item Prefisso: $/26$. Netmask: \texttt{255.255.255.192}.
    \item Allocazione IP1: \texttt{130.136.2.0 / 26}.
\end{itemize}
\begin{minted}{text}
Rete IP1:
Network:    130.136.2.0
Netmask:    255.255.255.192
First Host: 130.136.2.1
Last Host:  130.136.2.61
Router:     130.136.2.62
Broadcast:  130.136.2.63
\end{minted}

\subsubsection{Sottorete IP3 (4 hosts)}
Prossimo blocco dopo IP1 (che finisce a \texttt{130.136.2.63}).
\begin{itemize}
    \item Host richiesti: 4. Indirizzi totali: $4 + 2 = 6$.
    \item Potenza di 2 $\geq 6$: $8 = 2^3$. $h=3$ bit per host.
    \item Prefisso: $/29$. Netmask: \texttt{255.255.255.248}.
    \item Allocazione IP3: \texttt{130.136.2.64 / 29}.
\end{itemize}
\begin{minted}{text}
Rete IP3:
Network:    130.136.2.64
Netmask:    255.255.255.248
First Host: 130.136.2.65
Last Host:  130.136.2.69
Router:     130.136.2.70
Broadcast:  130.136.2.71
\end{minted}

\subsection{Considerazioni sull'Esempio 2 (Default Routers)}
\begin{itemize}
    \item Host in IP2-B (\texttt{130.136.0.0/24}) usano \texttt{130.136.0.254} come gateway.
    \item Host in IP2-A (\texttt{130.136.1.0/25}) usano \texttt{130.136.1.126} come gateway.
    \item Host in IP2 ma fuori da IP2-A/IP2-B (range \texttt{130.136.1.128} - \texttt{130.136.1.253}, con netmask \texttt{/23}) usano \texttt{130.136.1.254} come gateway.
    \item Il router \texttt{130.136.0.254} (di IP2-B) e \texttt{130.136.1.126} (di IP2-A) avranno come default gateway \texttt{130.136.1.254} (router di IP2).
    \item Host in IP1 (\texttt{130.136.2.0/26}) usano \texttt{130.136.2.62} come gateway.
    \item Host in IP3 (\texttt{130.136.2.64/29}) usano \texttt{130.136.2.70} come gateway.
    \item I router di IP1, IP2, IP3 avranno come default gateway il router principale della rete \texttt{130.136.0.0/16} (es. \texttt{130.136.255.254}, se così configurato, o un router del provider).
\end{itemize}

\newpage
\section{Riepilogo Punti Chiave}
\begin{itemize}
    \item Il subnetting e VLSM sono essenziali per un uso efficiente degli indirizzi IP.
    \item Si parte sempre dal numero di host richiesti, si aggiungono 2 (indirizzo di rete e di broadcast), e si trova la potenza di 2 immediatamente superiore o uguale per determinare i bit di host necessari.
    \item L'allocazione dei blocchi di indirizzi dovrebbe essere contigua per evitare la frammentazione dello spazio di indirizzamento e semplificare la gestione.
    \item Ogni sottorete definita ha il suo indirizzo di rete, il suo indirizzo di broadcast e necessita di un router (gateway) per la comunicazione esterna.
    \item La gerarchia delle reti implica che i router delle sottoreti più piccole puntino ai router delle reti che le contengono come loro default gateway, fino al router principale della rete o del provider.
    \item La comprensione della maschera di sottorete è cruciale per ogni host per determinare quali altri host sono nella sua stessa sottorete e quale indirizzo usare come broadcast.
\end{itemize}

\end{document}