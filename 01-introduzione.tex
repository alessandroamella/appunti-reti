%%%%%%%%%%%%%%%%%%%%%%%%%%%%%%%%%%%%%%%%%%%%%%%%%%%%%%%%%%%%%%%%%%%%%%%%%%%%%%%
% PREAMBOLO COMUNE PER APPUNTI (Stile Scuro)
%
% Questo file contiene tutte le impostazioni e i pacchetti comuni.
% NON contiene \begin{document} o \end{document}.
%
% Istruzioni per la compilazione del file principale:
% pdflatex -shell-escape nomefile_principale.tex
%%%%%%%%%%%%%%%%%%%%%%%%%%%%%%%%%%%%%%%%%%%%%%%%%%%%%%%%%%%%%%%%%%%%%%%%%%%%%%%

\documentclass{article}

% --- Encoding e lingua ---
\usepackage[utf8]{inputenc}
\usepackage[italian]{babel}

% --- Margini e layout ---
\usepackage{geometry}
\geometry{a4paper, margin=1in}

% --- Font sans-serif (come Helvetica) ---
\usepackage[scaled]{helvet}
\renewcommand{\familydefault}{\sfdefault}
\usepackage[T1]{fontenc}

% --- Matematica ---
\usepackage{amsmath}
\usepackage{amssymb}

% --- Liste personalizzate ---
\usepackage{enumitem}
% \setlist{nosep}

% --- Immagini e Grafica ---
\usepackage{float}
% \usepackage{graphicx}
\usepackage{tikz}
\usetikzlibrary{shapes.geometric, positioning, arrows.meta, calc, fit, backgrounds, patterns, decorations.pathreplacing}

% --- Tabelle Avanzate ---
\usepackage{array}
\usepackage{booktabs}
\usepackage{longtable}

% --- Hyperlink e Metadati PDF ---
\usepackage{hyperref}

\hypersetup{
    colorlinks=true,
    linkcolor=white,
    filecolor=magenta,
    urlcolor=cyan,
    citecolor=green,
    % pdftitle, pdfauthor, ecc. verranno impostati nel file principale
    pdfpagemode=FullScreen,
    bookmarksopen=true,
    bookmarksnumbered=true
}

% --- Licenza del documento ---
\usepackage[
  type={CC},
  modifier={by-sa},
  version={4.0},
]{doclicense}

% --- Colori e Sfondo Nero ---
\usepackage{xcolor}
\pagecolor{black}
\color{white}

% --- Evidenziazione del Codice ---
\usepackage{minted}
\setminted{
    frame=lines,
    framesep=2mm,
    fontsize=\small,
    breaklines=true,
    style=monokai,
    bgcolor=black!80
}
\usemintedstyle{monokai}

% --- Comandi personalizzati per algebra relazionale ---
\newcommand{\Rel}[1]{\textit{#1}} % Per i nomi delle relazioni
\newcommand{\Attr}[1]{\textsf{#1}} % Per i nomi degli attributi

\newcommand{\myunion}{\cup}
\newcommand{\myintersection}{\cap}
\newcommand{\mydifference}{-}
\newcommand{\myrename}[2]{\rho_{#1}(#2)}
\newcommand{\myselectop}[2]{\sigma_{#1}(#2)}
\newcommand{\myproject}[2]{\pi_{#1}(#2)}
\newcommand{\mycartesian}{\times}
\newcommand{\mynaturaljoin}{\bowtie} % Usare \Join da amssymb se disponibile e preferito
\newcommand{\mythetajoin}[3]{#1 \bowtie_{#2} #3} % R1 \bowtie_cond R2

% --- Comandi personalizzati per logica ---
\newcommand{\mylandop}{\wedge}
\newcommand{\myvel}{\vee}
\newcommand{\mynegop}{\neg}
\newcommand{\myforallop}{\forall}
\newcommand{\myexistsop}{\exists}

% --- Join esterni (outer join) ---
% Definizione standard per i join esterni
\def\ojoin{\setbox0=\hbox{$\mynaturaljoin$}%
	\rule[-.02ex]{.25em}{.4pt}\llap{\rule[\ht0]{.25em}{.4pt}}}
\newcommand{\myleftouterjoin}{\mathbin{\ojoin\mkern-5.8mu\mynaturaljoin}}
\newcommand{\myrightouterjoin}{\mathbin{\mynaturaljoin\mkern-5.8mu\ojoin}}
\newcommand{\myfullouterjoin}{\mathbin{\ojoin\mkern-5.8mu\mynaturaljoin\mkern-5.8mu\ojoin}}



\title{Introduzione}
\author{Basato sulle slide del Prof. Luciano Bononi}
\date{\today}

\begin{document}

\maketitle
\tableofcontents
\newpage

\section{Introduzione alle Reti di Calcolatori}

\textit{"NON CHI COMINCIA MA QUEL CHE PERSEVERA"} - Motto della Nave Scuola Amerigo Vespucci.

\subsection{Cos'è una Rete di Calcolatori?}
\begin{itemize}
    \item \textbf{Definizione:} Un insieme di \textbf{dispositivi di calcolo autonomi e interconnessi}.
    \begin{itemize}
        \item \textbf{Autonomi:} Ogni dispositivo (computer, server, smartphone, stampante di rete, etc.) può funzionare indipendentemente.
        \item \textbf{Interconnessi:} I dispositivi possono scambiare informazioni tra loro.
    \end{itemize}
    \item \textbf{Motivazioni (Perché creare reti?):}
    \begin{enumerate}
        \item \textbf{Supporto alla comunicazione tra utenti:}
        \begin{itemize}
            \item Permette nuovi servizi come e-mail, World Wide Web (\texttt{WWW}), messaggistica istantanea, videoconferenze.
            \item \textit{Esempio pratico:} Puoi inviare un'email a un amico o navigare su un sito web grazie alla rete.
        \end{itemize}
        \item \textbf{Comunicazione tra calcolatori elettronici (condivisione):}
        \begin{itemize}
            \item \textbf{Condivisione di informazione:} Accesso a database condivisi, pagine web ospitate su server.
            \item \textit{Esempio pratico:} Più utenti possono accedere e modificare un documento condiviso su Google Drive.
            \item \textbf{Condivisione di dispositivi e risorse:} Utilizzo di stampanti di rete, dischi di memorizzazione centralizzati.
            \item \textit{Esempio pratico:} In un ufficio, tutti possono stampare su un'unica stampante connessa alla rete.
            \item \textbf{Accesso a calcolatori remoti:} Possibilità di utilizzare la potenza di calcolo o i dati di un computer situato altrove.
            \item \textit{Esempio pratico:} Connettersi da casa al server dell'università per lanciare simulazioni.
            \item \textbf{Calcolo distribuito e sistemi scalabili:} Suddivisione di compiti complessi tra più macchine, possibilità di aumentare le capacità del sistema aggiungendo più dispositivi.
            \item \textit{Esempio pratico:} Motori di ricerca come Google usano migliaia di server per processare le tue ricerche velocemente.
        \end{itemize}
    \end{enumerate}
    \item \textit{Terminologia:} Quando si parla di "rete" o "rete di comunicazione" in questo contesto, si intende implicitamente una rete di calcolatori elettronici. Reti telefoniche tradizionali o reti di distribuzione TV non sono considerate reti di calcolatori.
\end{itemize}

\subsection{Classificazione delle Reti (in base alla dimensione geografica)}
Le reti si possono classificare in base all'area geografica che coprono:
\begin{enumerate}
    \item \textbf{Reti Personali (Personal Area Network - \texttt{PAN}):}
    \begin{itemize}
        \item Connessione tra dispositivi molto vicini, tipicamente entro una stanza o sulla scrivania di una persona.
        \item \textit{Esempio pratico:} Connettere lo smartphone agli auricolari Bluetooth, o il laptop a una stampante wireless sulla stessa scrivania.
        \item Solitamente finanziate e gestite dal singolo utente.
    \end{itemize}
    \item \textbf{Reti Locali (Local Area Network - \texttt{LAN}):}
    \begin{itemize}
        \item Connessione di dispositivi all'interno di un ufficio, un laboratorio, un edificio o un campus universitario.
        \item Raggio di qualche centinaio di metri.
        \item \textit{Esempio pratico:} La rete Wi-Fi di casa tua, la rete cablata del dipartimento universitario.
        \item Spesso gestite da organizzazioni, università, aziende.
    \end{itemize}
    \item \textbf{Reti Metropolitane (Metropolitan Area Network - \texttt{MAN}):}
    \begin{itemize}
        \item Connessione su un'area urbana (una città).
        \item Raggio di decine di chilometri.
        \item \textit{Esempio pratico:} Una rete che collega diverse sedi di un'azienda sparse in una città, o reti civiche.
        \item Mantenute da provider di servizi di comunicazione o gestori telefonici.
    \end{itemize}
    \item \textbf{Reti Geografiche (Wide Area Network - \texttt{WAN}):}
    \begin{itemize}
        \item Connessione su aree molto ampie, nazionali, internazionali, o addirittura planetarie.
        \item \textit{Esempio pratico:} Internet è l'esempio più grande di \texttt{WAN}. Una multinazionale che collega i suoi uffici in diversi continenti usa una \texttt{WAN}.
        \item Struttura complessa, tecnologie eterogenee (cavi, fibra, satelliti). Gestite da enti nazionali/internazionali o grandi gestori.
    \end{itemize}
    \item \textbf{Internet:}
    \begin{itemize}
        \item È la \textbf{rete globale} composta dall'unione di moltissime reti di vari tipi (\texttt{PAN}, \texttt{LAN}, \texttt{MAN}, \texttt{WAN}) interconnesse.
        \item Queste reti comunicano tra loro usando un insieme comune di regole, detti \textbf{protocolli di Internet} (come \texttt{TCP/IP}). È una "rete di reti".
    \end{itemize}
\end{enumerate}

\subsection{Evoluzione e Costi delle Reti}
\begin{itemize}
    \item \textbf{Nascita di Internet:} Storicamente, la prima rete antesignana di Internet (\texttt{ARPANET}) nasce nel 1969 connettendo solo 4 calcolatori di 4 università americane.
    \item \textbf{Crescita:}
    \begin{itemize}
        \item Inizio 2003: oltre 172 milioni di calcolatori connessi (\textit{fonte: ISC}).
        \item Oggi (2020): stima di oltre 4 miliardi di dispositivi con indirizzo \texttt{IP} e oltre 25 miliardi considerando l'Internet of Things (\texttt{IoT}).
        \item Entro 2022-2025: previsione di oltre 60 miliardi di dispositivi \texttt{IoT} connessi.
        \item \textit{Nota:} Il numero di "host \texttt{DNS}" è un indice, ma non cattura tutti i dispositivi connessi. La crescita reale è ancora più esponenziale.
    \end{itemize}
    \item \textbf{Costi (chi paga?):}
    \begin{itemize}
        \item La realizzazione e gestione delle infrastrutture ha costi elevati, distribuiti tra molte entità.
        \item \textbf{Reti piccole (\texttt{LAN}):} Costi e gestione limitati, spesso locali.
        \item \textbf{Reti grandi (\texttt{MAN}, \texttt{WAN}, Internet):} Investimenti consistenti da parte di consorzi, provider nazionali/multinazionali.
        \item La maggior parte dell'infrastruttura di Internet è gestita capillarmente da piccoli gestori.
        \item \textbf{Costo per l'utente:} Tariffe a tempo, a quantità di dati, o "tutto incluso".
    \end{itemize}
\end{itemize}

\subsection{Prestazioni delle Reti}
Due aspetti principali di interesse per l'utente:
\begin{enumerate}
    \item \textbf{Capacità di Trasmissione (Bandwidth):}
    \begin{itemize}
        \item Numero di bit o byte trasmessi/ricevuti in un secondo.
        \item Impropriamente detta "velocità della rete".
        \item Unità di misura:
        \begin{itemize}
            \item \textbf{bit} (b) o \textbf{byte} (B) (1 Byte = 8 bit).
            \item Prefissi: Kilo (K, migliaia), Mega (M, milioni), Giga (G, miliardi), Tera (T, migliaia di miliardi).
            \item Es: 10 Mbyte/sec = 10 milioni di gruppi da 8 bit al secondo.
        \end{itemize}
        \item \textit{Analogia:} Pensa a un tubo. La capacità è il diametro del tubo.
        \item Migliori tecnologie (es. fibra ottica) offrono capacità maggiori, ma a costi più elevati.
    \end{itemize}
    \item \textbf{Ritardo del Collegamento (Latency/Delay):}
    \begin{itemize}
        \item Tempo richiesto ai dati per transitare da mittente a destinatario.
        \item Dipende da: Distanza fisica, Tempi di gestione dei protocolli.
        \item \textit{Analogia:} La lunghezza del tubo.
        \item \textit{Esempio confronto:} Trasportare una grande quantità di dati.
        \begin{itemize}
            \item Rete \texttt{GARR} (1 GB/s): Potrebbe trasferire 10800 GB in 3 ore.
            \item Furgone da Milano a Roma (3 ore): Potrebbe trasportare 11000 dischi da 1GB (11000 GB).
            \item \textit{Quale "rete" è migliore?} Dipende. Per comunicazioni interattive, basso ritardo è cruciale.
        \end{itemize}
    \end{itemize}
\end{enumerate}
Altri indici di prestazione, dipendenti dall'applicazione:
\begin{itemize}
    \item \textbf{Jitter (Variazione del Ritardo):}
    \begin{itemize}
        \item È lo "spargimento" del ritardo di arrivo dei pacchetti attorno a un valore medio.
        \item Critico per applicazioni di streaming (audio/video).
        \item \textit{Esempio pratico:} Video online che si blocca o "scatta".
    \end{itemize}
    \item \textbf{\texttt{RTT} (Round Trip Time):}
    \begin{itemize}
        \item Ritardo di andata e ritorno.
        \item Critico per applicazioni interattive come i giochi online.
        \item \textit{Esempio pratico:} In un gioco sparatutto, un \texttt{RTT} alto causa svantaggio.
    \end{itemize}
\end{itemize}

\subsection{Il Calcolatore e la Rete: Componenti}
Per connettere un calcolatore a una rete servono componenti hardware e software aggiuntivi:
\begin{enumerate}
    \item \textbf{Dispositivi o Schede di Rete (Hardware):}
    \begin{itemize}
        \item Componente fisico (es. scheda Ethernet, scheda Wi-Fi).
        \item Funzioni: codificare/decodificare dati in segnali e viceversa.
        \item Amministrati da software del sistema operativo.
        \item Ogni scheda di rete ha un \textbf{indirizzo \texttt{MAC} (Medium Access Control)}: codice identificativo unico mondiale.
        \item \textit{Esempio pratico:} Come il numero di telaio di un'automobile.
    \end{itemize}
    \item \textbf{Mezzo di Trasmissione (Hardware):}
    \begin{itemize}
        \item Supporto fisico per la propagazione dei segnali.
        \item Tipi:
        \begin{itemize}
            \item \textbf{Cavetti o fili metallici:} Trasmettono segnali elettrici (es. cavo Ethernet).
            \item \textbf{Fibre Ottiche:} Trasmettono segnali luminosi. Prestazioni elevate.
            \item \textbf{Senza Fili (Wireless):} Usano onde elettromagnetiche. Permettono mobilità.
        \end{itemize}
    \end{itemize}
    \item \textbf{Connettore di Rete (Hardware):}
    \begin{itemize}
        \item Interfaccia standard per collegare dispositivo di rete al mezzo di trasmissione.
        \item \textit{Esempio pratico:} Porta \texttt{RJ45} per cavi Ethernet.
    \end{itemize}
    \item \textbf{Protocolli di Rete (Software):}
    \begin{itemize}
        \item Insieme di regole implementate come software.
        \item Definiscono come avviene la comunicazione per garantire compatibilità.
        \item \textit{Esempio pratico:} Due persone che parlano la stessa lingua e seguono regole di cortesia.
    \end{itemize}
\end{enumerate}

\subsection{Collegamenti e Infrastrutture di Rete}
\begin{itemize}
    \item \textbf{Connessione/Collegamento di Rete:} Mezzo di trasmissione condiviso.
    \item \textbf{Infrastruttura di Rete:} Insieme dei collegamenti fisici.
    \item \textbf{Cammino dei Segnali:} Percorso dei segnali.
    \item \textbf{Classi di Strutture di Connessione (Topologie Semplici):}
    \begin{itemize}
        \item \textbf{(a) Punto a Punto:} Connessione diretta tra due dispositivi.
        \item \textbf{(b) Completamente Connessa (Full Mesh):} Ogni dispositivo connesso a tutti gli altri. Ridondante ma costosa.
        \item \textbf{(c) Parzialmente Connessa (Partial Mesh):} Garantisce almeno un cammino tra ogni coppia.
        \item \textbf{(d) Partizioni di Rete:} Gruppi isolati.
    \end{itemize}
    \item \textbf{Topologie di Rete (Schemi di Connessione):}
    \begin{itemize}
        \item \textbf{(a) Anello (Ring):} Ogni dispositivo connesso al precedente e al successivo.
        \item \textbf{(b) Stella (Star):} Dispositivi periferici connessi a un dispositivo centrale.
        \item \textbf{(c) Bus:} Tutti i dispositivi connessi a un canale condiviso.
        \item \textbf{(d) Albero (Tree):} Struttura gerarchica.
        \item \textit{Nota:} Topologie semplici comuni in \texttt{LAN}/\texttt{PAN}. Reti più grandi hanno topologie complesse a grafo (maglia).
    \end{itemize}
\end{itemize}

\subsection{Canali di Comunicazione della Rete}
\begin{itemize}
    \item \textbf{Canale di Comunicazione:} Un "tubo virtuale" sul mezzo fisico.
    \item \textbf{Canale Punto a Punto:} Riservato tra due soli dispositivi.
    \item \textbf{Canale ad Accesso Multiplo (Canale Broadcast):}
    \begin{itemize}
        \item Tutti i dispositivi trasmettono e ricevono sullo stesso canale condiviso.
        \item \textbf{Problemi:}
        \begin{enumerate}
            \item \textbf{Arbitraggio:} Chi trasmette? E quando?
            \item \textbf{Rischio Collisione:} Se due o più dispositivi trasmettono simultaneamente, i segnali si distruggono.
            \item \textit{Esempio pratico:} Più persone che parlano contemporaneamente.
            \item \textbf{Indirizzamento:} A chi sono destinati i dati?
        \end{enumerate}
    \end{itemize}
\end{itemize}

\subsection{Reti a Commutazione di Circuito vs. Pacchetto}
Modi per trasferire informazioni:
\begin{enumerate}
    \item \textbf{Commutazione di Circuito (Circuit Switching):}
    \begin{itemize}
        \item Viene \textbf{riservato un circuito} (cammino dedicato) prima e per tutta la trasmissione.
        \item \textit{Esempio classico:} Telefonata tradizionale.
        \item \textbf{Caratteristiche:} Ritardo basso e costante; si paga il tempo di connessione; basso utilizzo risorse (spreco).
    \end{itemize}
    \item \textbf{Commutazione di Pacchetto (Packet Switching):}
    \begin{itemize}
        \item I dati sono suddivisi in \textbf{pacchetti}.
        \item Ogni pacchetto è spedito \textbf{indipendentemente}. Canali condivisi.
        \item \textit{Esempio classico:} Email, navigare sul web.
        \item \textbf{Caratteristiche:} Ogni pacchetto necessita indirizzo destinatario; migliore utilizzo risorse; ritardo maggiore e variabile; si paga per quantità di dati.
    \end{itemize}
\end{enumerate}

\subsection{Servizi Orientati alla Connessione e Non (reti a commutazione di pacchetto)}
\begin{enumerate}
    \item \textbf{Servizi Orientati alla Connessione (Connection-Oriented):}
    \begin{itemize}
        \item Stabilita una "connessione logica" (circuito virtuale) prima della trasmissione.
        \item Obiettivo: comunicazione che appare affidabile e ordinata.
        \item \textit{Esempio:} Telefonata \texttt{VoIP}, streaming video, download file con \texttt{TCP}.
        \item Garantiscono (o tentano): consegna ordinata, ri-trasmissione pacchetti persi.
    \end{itemize}
    \item \textbf{Servizi Non Orientati alla Connessione (Connectionless):}
    \begin{itemize}
        \item Nessuna connessione preliminare. Ogni pacchetto trattato indipendentemente.
        \item \textit{Esempio:} Posta ordinaria. In reti: \texttt{DNS}, alcuni giochi online, streaming con \texttt{UDP}.
        \item Caratteristiche: pacchetti possono seguire strade diverse, arrivare fuori ordine o perdersi; meno overhead.
    \end{itemize}
\end{enumerate}

\subsection{Protocolli di Rete Organizzati a Livelli (Architettura a Livelli)}
\begin{itemize}
    \item \textbf{Protocollo:} Insieme di regole (semantiche e sintattiche) per lo scambio di messaggi.
    \begin{itemize}
        \item \textbf{Semantica:} Significato dei messaggi.
        \item \textbf{Sintassi:} Formato dei messaggi.
        \item Permettono compatibilità tra sistemi eterogenei.
    \end{itemize}
    \item \textbf{Architettura dei Protocolli di Rete:} Struttura dei livelli.
    \begin{itemize}
        \item Ogni livello risolve un sottoinsieme di problemi.
        \item Ogni livello offre \textbf{servizi} al livello superiore e usa i servizi del livello inferiore.
        \item \textbf{Interfaccia tra livelli:} Definisce come un livello interagisce con quelli adiacenti.
        \item \textbf{Comunicazione tra pari (Peer-to-Peer):} Un livello \texttt{X} su una macchina comunica virtualmente con il livello \texttt{X} su un'altra.
    \end{itemize}
    \item \textit{Esempio Pratico (Innamorati Italiano e Giapponese):}
    Illustra come diversi "livelli" di servizio (dialogo, traduzione, dattilografia, FAX) collaborano per raggiungere l'obiettivo finale, ognuno gestendo un aspetto specifico del problema.
\end{itemize}

\subsection{Architettura Standard di Protocolli di Rete: \texttt{ISO/OSI RM}}
\begin{itemize}
    \item \textbf{Standard \texttt{ISO/OSI} Reference Model:} Modello teorico di riferimento con \textbf{7 livelli}.
    \item \textbf{Scopo:} Standard per comunicazione tra sistemi aperti.
    \item \textbf{I Sette Livelli \texttt{ISO/OSI} (dall'alto verso il basso):}
    \begin{enumerate}[label=\arabic*.]
        \setcounter{enumi}{6} % Start numbering from 7
        \item \textbf{Livello 7 - Applicazione:} Interfaccia per applicazioni utente (es. \texttt{HTTP}, \texttt{SMTP}).
        \item \textbf{Livello 6 - Presentazione:} Gestione sintassi/semantica dati (traduzione, compressione, crittografia).
        \item \textbf{Livello 5 - Sessione:} Stabilisce, gestisce, termina sessioni di comunicazione.
        \item \textbf{Livello 4 - Trasporto:} Trasferimento dati affidabile end-to-end (\texttt{TCP}, \texttt{UDP}). Controllo congestione.
        \item \textbf{Livello 3 - Rete:} Instradamento (routing) pacchetti attraverso la rete (\texttt{IP}). Indirizzamento logico.
        \item \textbf{Livello 2 - Data Link:} Trasferimento affidabile frame tra nodi direttamente connessi. Indirizzamento fisico (\texttt{MAC}). Controllo accesso.
        \item \textbf{Livello 1 - Fisico:} Trasmissione bit grezzi sul mezzo fisico. Specifiche elettriche/meccaniche.
    \end{enumerate}
\end{itemize}

\subsection{Architettura dei Protocolli di Internet (\texttt{TCP/IP} Stack)}
\begin{itemize}
    \item Architettura semplificata dell'\texttt{ISO/OSI}, tipicamente con \textbf{5 livelli}.
    \item \textbf{Livelli Internet (dal basso verso l'alto):}
    \begin{enumerate}
        \item \textbf{Fisico}
        \item \textbf{Data Link / \texttt{MAC/LLC}}
        \item \textbf{Rete (Network / Internet)} (protocollo principale: \texttt{IP})
        \item \textbf{Trasporto (Transport)} (protocolli principali: \texttt{TCP}, \texttt{UDP})
        \item \textbf{Applicazione} (comprende funzioni dei livelli 5, 6, 7 di \texttt{ISO/OSI})
    \end{enumerate}
    \item \textbf{Incapsulamento (Encapsulation):}
    \begin{itemize}
        \item \textbf{In trasmissione:} Ogni livello aggiunge info di controllo (header/trailer) ai dati dal livello superiore.
        \begin{itemize}
            \item Applicazione $\rightarrow$ dati
            \item Trasporto $\rightarrow$ \textit{segmento/datagramma} (\texttt{TCP/UDP} header + dati)
            \item Rete $\rightarrow$ \textit{pacchetto} (\texttt{IP} header + segmento/datagramma)
            \item Data Link $\rightarrow$ \textit{frame} (\texttt{MAC} header + pacchetto + \texttt{MAC} trailer)
            \item Fisico $\rightarrow$ bit
        \end{itemize}
        \item \textbf{In ricezione:} Processo inverso (\textbf{decapsulamento}).
    \end{itemize}
\end{itemize}

\subsection{Confronto Visivo tra Modelli \texttt{ISO/OSI} e \texttt{TCP/IP}}
\begin{figure}[H]
\centering
\begin{tikzpicture}[
    box/.style={rectangle, draw=primarytext, fill=gray!20, text=black, 
               minimum width=3.5cm, minimum height=0.7cm, text centered},
    arrow/.style={->, >=Stealth, thick, draw=primarytext},
    bracket/.style={decorate, decoration={brace, amplitude=5pt}, thick, draw=primarytext},
    note/.style={text=darktext, font=\footnotesize\itshape}
]

% ISO/OSI Model (left side)
\node[box, fill=accentcolor!20] at (0,0) {1. Fisico};
\node[box, fill=accentcolor!30] at (0,0.9) {2. Data Link};
\node[box, fill=accentcolor!40] at (0,1.8) {3. Rete};
\node[box, fill=accentcolor!50] at (0,2.7) {4. Trasporto};
\node[box, fill=accentcolor!60] at (0,3.6) {5. Sessione};
\node[box, fill=accentcolor!70] at (0,4.5) {6. Presentazione};
\node[box, fill=accentcolor!80] at (0,5.4) {7. Applicazione};

\node[note] at (0,-0.7) {\textbf{Modello ISO/OSI}};

% TCP/IP Model (right side)
\node[box, fill=highlightcolor!20] at (6,0) {1. Fisico};
\node[box, fill=highlightcolor!30] at (6,0.9) {2. Data Link / MAC/LLC};
\node[box, fill=highlightcolor!40] at (6,1.8) {3. Rete (IP)};
\node[box, fill=highlightcolor!50] at (6,2.7) {4. Trasporto (TCP/UDP)};
\node[box, fill=highlightcolor!80] at (6,4.5) {5. Applicazione};

\node[note] at (6,-0.7) {\textbf{Stack TCP/IP}};

% Mapping lines between models
\draw[arrow, dashed] (1.8,0) -- (4.2,0);
\draw[arrow, dashed] (1.8,0.9) -- (4.2,0.9);
\draw[arrow, dashed] (1.8,1.8) -- (4.2,1.8);
\draw[arrow, dashed] (1.8,2.7) -- (4.2,2.7);

% Bracket for combined layers in TCP/IP
\draw[bracket] (1.8,5.4) -- (1.8,3.6);
\draw[arrow, dashed] (2.0,4.5) -- (4.2,4.5);

% Data flow arrows (right side)
\draw[arrow, orange, ultra thick] (8.5,5.2) -- (8.5,4.8) node[right, midway, text=primarytext] {Dati};
\draw[arrow, orange, ultra thick] (8.5,4.2) -- (8.5,3.8) node[right, midway, text=primarytext] {\textit{Segmento}};
\draw[arrow, orange, ultra thick] (8.5,3.3) -- (8.5,2.9) node[right, midway, text=primarytext] {\textit{Pacchetto}};
\draw[arrow, orange, ultra thick] (8.5,2.4) -- (8.5,2.0) node[right, midway, text=primarytext] {\textit{Frame}};
\draw[arrow, orange, ultra thick] (8.5,1.5) -- (8.5,1.1) node[right, midway, text=primarytext] {Bit};

% Add encapsulation labels
\node[note] at (8.5,5.8) {Incapsulamento};
\node[note] at (9.8,3.6) {$\leftarrow$ Header};

% Protocol examples (optional) - Moving them further to the left
\node[note] at (-3.2,5.4) {HTTP, SMTP, FTP};
\node[note] at (-3.2,4.5) {SSL/TLS, JPEG};
\node[note] at (-3.2,3.6) {NetBIOS};
\node[note] at (-3.2,2.7) {TCP, UDP};
\node[note] at (-3.2,1.8) {IP, ICMP};
\node[note] at (-3.2,0.9) {Ethernet, 802.11};
\node[note] at (-3.2,0) {RS-232, Bluetooth};

\end{tikzpicture}
\caption{Confronto tra modello \texttt{ISO/OSI} e stack \texttt{TCP/IP} con flusso di incapsulamento dati}
\end{figure}

\subsection{Livelli e Integrazione delle Reti (Prospettiva Funzionale)}

\begin{table}[H]
\centering
\begin{tabular}{|p{3cm}|p{12cm}|}
\hline
\rowcolor{bg_custom} \textbf{Livello} & \textbf{Visione della Rete} \\
\hline
Applicazione & Utilizzabile dalle applicazioni utente \\
\hline
Trasporto & Offre comunicazione \textbf{end-to-end} \\
\hline
Rete & Una collezione di reti interconnesse (\textbf{internetwork}). Disp: Router \\
\hline
\texttt{MAC/LLC} & Una rete locale (\texttt{LAN}). Può integrare tecnologie diverse. Disp: Switch, Bridge \\
\hline
Fisico & Un segmento di mezzo trasmissivo condiviso \\
\hline
\end{tabular}
\caption{Visione della rete ai diversi livelli}
\end{table}

\subsection{Dispositivi di Rete e Affidabilità}

\begin{table}[H]
\centering
\begin{tabular}{|p{3cm}|p{3cm}|p{9cm}|}
\hline
\rowcolor{bg_custom} \textbf{Dispositivo} & \textbf{Livello} & \textbf{Funzionalità} \\
\hline
Repeater & Livello 1 & Amplifica/rigenera segnale. Non capisce \texttt{MAC}. Estende dominio di collisione \\
\hline
Hub & Livello 1 & Repeater multiporta. Crea unico grande dominio di collisione \\
\hline
Bridge & Livello 2 & Connette segmenti \texttt{MAC} diversi. Filtra/inoltra frame per indirizzi \texttt{MAC} \\
\hline
Switch & Livello 2 & Bridge multiporta intelligente. Ogni porta è dominio separato \\
\hline
\end{tabular}
\caption{Dispositivi di rete di livello 1 e 2}
\end{table}

\begin{table}[H]
\centering
\begin{tabular}{|p{4cm}|p{11cm}|}
\hline
\rowcolor{bg_custom} \textbf{Tipo di Affidabilità} & \textbf{Caratteristiche} \\
\hline
\textbf{Livello 2} \newline (\texttt{MAC/LLC}) \newline hop-by-hop reliability & 
• Affidabilità su singolo collegamento fisico \newline
• Usa ACK e ritrasmissioni \newline
• Tentativo rapido e locale \\
\hline
\textbf{Livello 4} \newline (Trasporto, es. \texttt{TCP}) \newline end-to-end reliability & 
• Affidabilità tra applicazioni finali \newline
• Attraversa l'intera rete \newline
• Gestisce problemi globali (perdita pacchetti, congestione) \\
\hline
\end{tabular}
\caption{Tipi di affidabilità nella rete}
\end{table}

\subsection{Tecnologie per Schede di Rete}

\begin{table}[H]
\centering
\begin{tabular}{|>{\centering\arraybackslash}m{3.5cm}|>{\raggedright\arraybackslash}m{11.5cm}|}
\hline
\rowcolor{bg_custom} \textbf{Tecnologia} & \textbf{Caratteristiche} \\
\hline
\begin{tabular}[c]{@{}c@{}}\textbf{Ethernet}\\\texttt{(CSMA/CD)}\end{tabular} & 
\begin{itemize}[leftmargin=*]
    \item Molto usato in reti locali cablate
    \item \textbf{Carrier Sense:} Ascolta il canale prima di trasmettere
    \item \textbf{Multiple Access:} Più dispositivi condividono
    \item \textbf{Collision Detection:} Se rileva collisione, interrompe, invia jam, attende (backoff), ritenta
\end{itemize} \\[2pt]
\hline
\begin{tabular}[c]{@{}c@{}}\textbf{Wi-Fi}\\\texttt{(IEEE 802.11)}\\\texttt{(CSMA/CA)}\end{tabular} & 
\begin{itemize}[leftmargin=*]
    \item Reti locali senza fili
    \item Collision Detection difficile in wireless
    \item \textbf{Collision Avoidance:} Si cerca di \textbf{evitare} collisioni
    \item Usa intervalli DIFS, tempi casuali, ACK espliciti, \texttt{RTS/CTS}
\end{itemize} \\[2pt]
\hline
\begin{tabular}[c]{@{}c@{}}\textbf{Token Ring}\\(e Token Bus)\end{tabular} & 
\begin{itemize}[leftmargin=*]
    \item Meno comune oggi
    \item Accesso regolato da \textbf{token}
    \item \textbf{Vantaggi:} No collisioni, prestazioni deterministiche
    \item \textbf{Svantaggi:} Complessità, gestione token
\end{itemize} \\
\hline
\end{tabular}
\caption{Confronto tra tecnologie di rete di livello 2 (\texttt{MAC})}
\end{table}

\end{document}