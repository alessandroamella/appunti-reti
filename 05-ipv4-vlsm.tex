%%%%%%%%%%%%%%%%%%%%%%%%%%%%%%%%%%%%%%%%%%%%%%%%%%%%%%%%%%%%%%%%%%%%%%%%%%%%%%%
% PREAMBOLO COMUNE PER APPUNTI (Stile Scuro)
%
% Questo file contiene tutte le impostazioni e i pacchetti comuni.
% NON contiene \begin{document} o \end{document}.
%
% Istruzioni per la compilazione del file principale:
% pdflatex -shell-escape nomefile_principale.tex
%%%%%%%%%%%%%%%%%%%%%%%%%%%%%%%%%%%%%%%%%%%%%%%%%%%%%%%%%%%%%%%%%%%%%%%%%%%%%%%

\documentclass{article}

% --- Encoding e lingua ---
\usepackage[utf8]{inputenc}
\usepackage[italian]{babel}

% --- Margini e layout ---
\usepackage{geometry}
\geometry{a4paper, margin=1in}

% --- Font sans-serif (come Helvetica) ---
\usepackage[scaled]{helvet}
\renewcommand{\familydefault}{\sfdefault}
\usepackage[T1]{fontenc}

% --- Matematica ---
\usepackage{amsmath}
\usepackage{amssymb}

% --- Liste personalizzate ---
\usepackage{enumitem}
% \setlist{nosep}

% --- Immagini e Grafica ---
\usepackage{float}
% \usepackage{graphicx}
\usepackage{tikz}
\usetikzlibrary{shapes.geometric, positioning, arrows.meta, calc, fit, backgrounds, patterns, decorations.pathreplacing}

% --- Tabelle Avanzate ---
\usepackage{array}
\usepackage{booktabs}
\usepackage{longtable}

\usepackage{siunitx} % Per unità di misura come GHz, dBm, ecc.

% --- Hyperlink e Metadati PDF ---
\usepackage{hyperref}

\usepackage{graphicx}

\hypersetup{
    colorlinks=true,
    linkcolor=white,
    filecolor=magenta,
    urlcolor=cyan,
    citecolor=green,
    % pdftitle, pdfauthor, ecc. verranno impostati nel file principale
    pdfpagemode=FullScreen,
    bookmarksopen=true,
    bookmarksnumbered=true
}

% --- Licenza del documento ---
\usepackage[
  type={CC},
  modifier={by-sa},
  version={4.0},
]{doclicense}

% --- Colori e Sfondo Nero ---
\usepackage{xcolor}
\pagecolor{black}
\color{white}

% --- Evidenziazione del Codice ---
\usepackage{minted}
\setminted{
    frame=lines,
    framesep=2mm,
    fontsize=\small,
    breaklines=true,
    style=monokai,
    bgcolor=black!80
}
\usemintedstyle{monokai}

% --- Comandi personalizzati per algebra relazionale ---
\newcommand{\Rel}[1]{\textit{#1}} % Per i nomi delle relazioni
\newcommand{\Attr}[1]{\textsf{#1}} % Per i nomi degli attributi

\newcommand{\myunion}{\cup}
\newcommand{\myintersection}{\cap}
\newcommand{\mydifference}{-}
\newcommand{\myrename}[2]{\rho_{#1}(#2)}
\newcommand{\myselectop}[2]{\sigma_{#1}(#2)}
\newcommand{\myproject}[2]{\pi_{#1}(#2)}
\newcommand{\mycartesian}{\times}
\newcommand{\mynaturaljoin}{\bowtie} % Usare \Join da amssymb se disponibile e preferito
\newcommand{\mythetajoin}[3]{#1 \bowtie_{#2} #3} % R1 \bowtie_cond R2

% --- Comandi personalizzati per logica ---
\newcommand{\mylandop}{\wedge}
\newcommand{\myvel}{\vee}
\newcommand{\mynegop}{\neg}
\newcommand{\myforallop}{\forall}
\newcommand{\myexistsop}{\exists}

% --- Join esterni (outer join) ---
% Definizione standard per i join esterni
\def\ojoin{\setbox0=\hbox{$\mynaturaljoin$}%
	\rule[-.02ex]{.25em}{.4pt}\llap{\rule[\ht0]{.25em}{.4pt}}}
\newcommand{\myleftouterjoin}{\mathbin{\ojoin\mkern-5.8mu\mynaturaljoin}}
\newcommand{\myrightouterjoin}{\mathbin{\mynaturaljoin\mkern-5.8mu\ojoin}}
\newcommand{\myfullouterjoin}{\mathbin{\ojoin\mkern-5.8mu\mynaturaljoin\mkern-5.8mu\ojoin}}



\title{Appunti su Progettazione Reti IPv4 e Subnetting}
\author{Basato sulle slide del Prof. Luciano Bononi}
\date{\today}

\begin{document}

\maketitle
\tableofcontents
\newpage

\section{Introduzione alla Progettazione di Reti IPv4 e Subnetting VLSM}
L'obiettivo della progettazione di reti IPv4 è allocare in modo efficiente lo spazio di indirizzamento IP disponibile per soddisfare i requisiti di host di diverse sottoreti, minimizzando lo spreco di indirizzi. Questo approccio utilizza spesso il Variable Length Subnet Masking (VLSM).

\subsection{Concetti Fondamentali}
\begin{itemize}
    \item \textbf{Indirizzo di Rete (Network Address):} Il primo indirizzo di un blocco IP, con tutti i bit della parte host a 0. Non è assegnabile a un host. Esempio: \texttt{192.168.1.0}.
    \item \textbf{Indirizzo di Broadcast:} L'ultimo indirizzo di un blocco IP, con tutti i bit della parte host a 1. Invia dati a tutti gli host in quella specifica sottorete. Non è assegnabile a un host. Esempio: \texttt{192.168.1.255} per una rete /24.
    \item \textbf{Host Utilizzabili:} Gli indirizzi compresi tra l'indirizzo di rete e l'indirizzo di broadcast. Il numero di host utilizzabili è $(2^h) - 2$, dove $h$ è il numero di bit dedicati alla parte host.
    \item \textbf{Netmask (Subnet Mask):} Una sequenza di bit che distingue la parte di rete dalla parte host in un indirizzo IP. I bit a 1 indicano la parte di rete, i bit a 0 la parte host. Esempio: \texttt{255.255.255.0}.
    \item \textbf{Notazione CIDR (Classless Inter-Domain Routing):} Indica il numero di bit a 1 consecutivi nella netmask (cioè i bit della parte di rete). Esempio: \texttt{/24} corrisponde a \texttt{255.255.255.0}.
    \item \textbf{Router:} Un dispositivo che connette diverse reti o sottoreti. Ogni interfaccia del router ha un indirizzo IP che appartiene alla sottorete a cui è connessa e funge da gateway per gli host di quella sottorete. Solitamente si assegna il primo o l'ultimo indirizzo IP \textit{utilizzabile} della sottorete al router.
    \item \textbf{Router di Default (Default Gateway):} L'indirizzo IP del router che gli host di una sottorete usano per comunicare con host esterni alla loro sottorete.
\end{itemize}

\textbf{Esempio Visuale: Come Leggere un Indirizzo IP con CIDR}

% Definiamo alcuni colori specifici per la tabella
\definecolor{retecolor}{RGB}{100,150,255}
\definecolor{hostcolor}{RGB}{255,150,100}

\begin{table}[h]
\centering
\begin{tabular}{|l|l|}
\hline
\rowcolor{bg_custom}
\textcolor{white}{\textbf{Componente}} & \textcolor{white}{\textbf{Valore}} \\
\hline
\textbf{Indirizzo IP} & \texttt{192.168.1.64} \\
\hline
\textbf{Prefisso CIDR} & \texttt{/26} (26 bit per la rete, 6 bit per gli host) \\
\hline
\textbf{Subnet Mask} & \texttt{255.255.255.192} \\
\hline
\textcolor{retecolor}{\textbf{Rete} (26 bit)} / \textcolor{hostcolor}{\textbf{Host} (6 bit)} & \textcolor{retecolor}{\texttt{11000000.10101000.00000001.01}}\textcolor{hostcolor}{\texttt{000000}} \\
\hline
\textbf{Indirizzo di Rete} & \texttt{192.168.1.64} (tutti i bit host = 0) \\
\hline
\textbf{Indirizzo di Broadcast} & \texttt{192.168.1.127} (tutti i bit host = 1) \\
\hline
\textbf{Range Indirizzi per Host} & \texttt{192.168.1.65} - \texttt{192.168.1.126} \\
\hline
\textbf{Numero Host Utilizzabili} & $2^6 - 2 = 64 - 2 = 62$ host \\
\hline
\textbf{Router/Gateway} & \texttt{192.168.1.126} (convenzionalmente l'ultimo) \\
\hline
\end{tabular}
\caption{Esempio: 192.168.1.64/26}
\end{table}

\vspace{1em}

\textbf{Confronto tra Diverse Dimensioni di Sottoreti}

\begin{table}[H]
\centering
\begin{tabular}{|c|c|c|c|c|c|}
\hline
\rowcolor{bg_custom}
\textcolor{white}{\textbf{CIDR}} & \textcolor{white}{\textbf{Subnet Mask}} & \textcolor{white}{\textbf{Bit Host}} & \textcolor{white}{\textbf{Tot. Indirizzi}} & \textcolor{white}{\textbf{Host Utili}} & \textcolor{white}{\textbf{Uso Tipico}} \\
\hline
\texttt{/24} & \texttt{255.255.255.0} & 8 & 256 & 254 & LAN media \\
\hline
\rowcolor{retecolor!20}
\textcolor{black}{\texttt{/25}} & \textcolor{black}{\texttt{255.255.255.128}} & \textcolor{black}{7} & \textcolor{black}{128} & \textcolor{black}{126} & \textcolor{black}{LAN piccola} \\
\hline
\texttt{/26} & \texttt{255.255.255.192} & 6 & 64 & 62 & Ufficio \\
\hline
\rowcolor{hostcolor!20}
\textcolor{black}{\texttt{/27}} & \textcolor{black}{\texttt{255.255.255.224}} & \textcolor{black}{5} & \textcolor{black}{32} & \textcolor{black}{30} & \textcolor{black}{Dipartimento} \\
\hline
\texttt{/28} & \texttt{255.255.255.240} & 4 & 16 & 14 & Piccolo gruppo \\
\hline
\rowcolor{retecolor!20}
\textcolor{black}{\texttt{/29}} & \textcolor{black}{\texttt{255.255.255.248}} & \textcolor{black}{3} & \textcolor{black}{8} & \textcolor{black}{6} & \textcolor{black}{Link P2P} \\
\hline
\texttt{/30} & \texttt{255.255.255.252} & 2 & 4 & 2 & Connessione diretta \\
\hline
\end{tabular}
\caption{Confronto tra diverse maschere di sottorete}
\end{table}

\vspace{1em}

\textbf{Esempio Dettagliato: Rete 10.0.0.0/8 Suddivisa}

\begin{table}[h]
\centering
\begin{tabular}{|l|c|c|c|c|}
\hline
\rowcolor{bg_custom}
\textcolor{white}{\textbf{Sottorete}} & \textcolor{white}{\textbf{Indirizzo}} & \textcolor{white}{\textbf{Range Host}} & \textcolor{white}{\textbf{Broadcast}} & \textcolor{white}{\textbf{Gateway}} \\
\hline
\rowcolor{retecolor!30}
\textcolor{black}{Rete A (/16)} & \textcolor{black}{\texttt{10.1.0.0}} & \textcolor{black}{\texttt{10.1.0.1 - 10.1.255.254}} & \textcolor{black}{\texttt{10.1.255.255}} & \textcolor{black}{\texttt{10.1.255.254}} \\
\hline
\rowcolor{hostcolor!30}
\textcolor{black}{Rete B (/16)} & \textcolor{black}{\texttt{10.2.0.0}} & \textcolor{black}{\texttt{10.2.0.1 - 10.2.255.254}} & \textcolor{black}{\texttt{10.2.255.255}} & \textcolor{black}{\texttt{10.2.255.254}} \\
\hline
\rowcolor{retecolor!30}
\textcolor{black}{Rete C (/24)} & \textcolor{black}{\texttt{10.3.1.0}} & \textcolor{black}{\texttt{10.3.1.1 - 10.3.1.254}} & \textcolor{black}{\texttt{10.3.1.255}} & \textcolor{black}{\texttt{10.3.1.254}} \\
\hline
\rowcolor{hostcolor!30}
\textcolor{black}{Rete D (/28)} & \textcolor{black}{\texttt{10.4.1.16}} & \textcolor{black}{\texttt{10.4.1.17 - 10.4.1.30}} & \textcolor{black}{\texttt{10.4.1.31}} & \textcolor{black}{\texttt{10.4.1.30}} \\
\hline
\end{tabular}
\caption{Suddivisione di 10.0.0.0/8 in sottoreti}
\end{table}

\subsection{Processo di Subnetting con VLSM}
VLSM permette di usare maschere di sottorete di lunghezza variabile per diverse sottoreti, ottimizzando l'uso degli indirizzi. La strategia generale è:
\begin{enumerate}
    \item \textbf{Elencare i Requisiti:} Determinare il numero di host necessari per ogni sottorete.
    \item \textbf{Ordinare per Dimensione:} Ordinare le sottoreti dalla più grande (più host richiesti) alla più piccola. Questo aiuta a prevenire la frammentazione dello spazio di indirizzamento.
    \item \textbf{Calcolare la Dimensione del Blocco:} Per ogni sottorete, calcolare il numero minimo di indirizzi totali necessari: $N_{\text{host richiesti}} + 2$ (per indirizzo di rete e broadcast). Trovare la più piccola potenza di 2 ($2^h$) che sia maggiore o uguale a questo valore. $h$ sarà il numero di bit per la parte host.
    \item \textbf{Determinare la Netmask/Prefisso:} Se $h$ sono i bit di host, allora $32 - h$ sono i bit di rete (il prefisso CIDR). Da qui si deriva la netmask.
    \item \textbf{Allocare gli Indirizzi:} Iniziare ad allocare blocchi di indirizzi contigui dal range disponibile, partendo dalla sottorete più grande.
\end{enumerate}

\newpage
\section{Esempio 1: Rete N: \texttt{199.201.17.0 / 24}}
Si dispone di un blocco iniziale \texttt{199.201.17.0 / 24}.

\textbf{Analisi del Blocco Iniziale:}
\begin{itemize}
    \item \textbf{Prefisso CIDR:} \texttt{/24} $\rightarrow$ 24 bit per la rete, 8 bit per gli host
    \item \textbf{Netmask:} \texttt{255.255.255.0} (equivalente di /24)
    \item \textbf{Indirizzi totali:} $2^8 = 256$ (da \texttt{.0} a \texttt{.255})
    \item \textbf{Host utilizzabili:} $2^8 - 2 = 254$ (esclusi network e broadcast)
\end{itemize}

\begin{table}[h]
\centering
\begin{tabular}{|l|c|l|}
\hline
\rowcolor{bg_custom}
\textcolor{white}{\textbf{Elemento}} & \textcolor{white}{\textbf{Indirizzo}} & \textcolor{white}{\textbf{Spiegazione}} \\
\hline
\textbf{Indirizzo di Rete N} & \texttt{199.201.17.0} & Tutti i bit host = 0 \\
\hline
\textbf{Primo Host N} & \texttt{199.201.17.1} & Network + 1 \\
\hline
\textbf{Ultimo Host N} & \texttt{199.201.17.253} & Broadcast - 2 (riservato per router) \\
\hline
\textbf{Router N (Gateway)} & \texttt{199.201.17.254} & Convenzionalmente penultimo indirizzo \\
\hline
\textbf{Broadcast N} & \texttt{199.201.17.255} & Tutti i bit host = 1 \\
\hline
\end{tabular}
\caption{Struttura della rete iniziale N}
\end{table}

\subsection{Suddivisione}
La suddivisione considera la gerarchia delle reti e alloca blocchi contigui.

\begin{figure}[H]
    \centering
    \includegraphics[width=0.7\textwidth]{images/subnet_pie_chart.png}
    \caption{Visualizzazione a torta della suddivisione dello spazio di indirizzi della rete 199.201.17.0/24}
    \label{fig:subnet_pie}
\end{figure}
    
\begin{figure}[H]
    \centering
    \includegraphics[width=1\textwidth]{images/subnet_linear_diagram.png}
    \caption{Diagramma lineare della suddivisione sequenziale delle subnet}
    \label{fig:subnet_linear}
\end{figure}
        

\subsubsection{Sottorete N2A (42 hosts)}
\begin{itemize}
    \item Host richiesti: 42. Indirizzi totali: $42 + 2 = 44$.
    \item Potenza di 2 $\geq 44$: $64 = 2^6$. Quindi $h=6$ bit per host.
    \item Prefisso: $32 - 6 = /26$. Netmask: \texttt{255.255.255.192}.
    \item Allocazione N2A: \texttt{199.201.17.0 / 26}.
\end{itemize}

\textbf{Calcolo della Netmask /26:}
\begin{itemize}
    \item 26 bit di rete = \texttt{11111111.11111111.11111111.11000000}
    \item In decimale: \texttt{255.255.255.192}
    \item Dimensione blocco: $2^6 = 64$ indirizzi (da \texttt{.0} a \texttt{.63})
\end{itemize}

\begin{table}[h]
\centering
\begin{tabular}{|l|c|l|}
\hline
\rowcolor{bg_custom}
\textcolor{white}{\textbf{Elemento}} & \textcolor{white}{\textbf{Indirizzo}} & \textcolor{white}{\textbf{Calcolo}} \\
\hline
\textbf{Network N2A} & \texttt{199.201.17.0} & Indirizzo base con /26 \\
\hline
\textbf{Netmask} & \texttt{255.255.255.192} & 26 bit a 1, 6 bit a 0 \\
\hline
\textbf{First Host} & \texttt{199.201.17.1} & Network + 1 \\
\hline
\textbf{Last Host} & \texttt{199.201.17.61} & Broadcast - 2 \\
\hline
\textbf{Router} & \texttt{199.201.17.62} & Broadcast - 1 \\
\hline
\textbf{Broadcast} & \texttt{199.201.17.63} & Network + $2^6 - 1$ = 0 + 63 \\
\hline
\end{tabular}
\caption{Struttura della sottorete N2A}
\end{table}

\subsubsection{Sottorete N2 (119 hosts, include N2A)}
\begin{itemize}
    \item Host richiesti (totali in N2): 119. Indirizzi totali: $119 + 2 = 121$.
    \item Potenza di 2 $\geq 121$: $128 = 2^7$. Quindi $h=7$ bit per host.
    \item Prefisso: $32 - 7 = /25$. Netmask: \texttt{255.255.255.128}.
    \item N2 ingloba N2A (\texttt{.0} - \texttt{.63}). Allocazione N2: \texttt{199.201.17.0 / 25}.
\end{itemize}

\textbf{Calcolo della Netmask /25:}
\begin{itemize}
    \item 25 bit di rete = \texttt{11111111.11111111.11111111.10000000}
    \item In decimale: \texttt{255.255.255.128}
    \item Dimensione blocco: $2^7 = 128$ indirizzi (da \texttt{.0} a \texttt{.127})
\end{itemize}

\begin{table}[h]
\centering
\begin{tabular}{|l|c|l|}
\hline
\rowcolor{bg_custom}
\textcolor{white}{\textbf{Elemento}} & \textcolor{white}{\textbf{Indirizzo}} & \textcolor{white}{\textbf{Note}} \\
\hline
\textbf{Network N2} & \texttt{199.201.17.0} & Stesso network di N2A ma /25 \\
\hline
\textbf{Netmask} & \texttt{255.255.255.128} & 25 bit a 1, 7 bit a 0 \\
\hline
\textbf{First Host} & \texttt{199.201.17.1} & Coincide con primo host N2A \\
\hline
\textbf{Last Host} & \texttt{199.201.17.125} & Broadcast - 2 \\
\hline
\textbf{Router N2} & \texttt{199.201.17.126} & Per host in N2 ma non in N2A \\
\hline
\textbf{Broadcast} & \texttt{199.201.17.127} & Network + $2^7 - 1$ = 0 + 127 \\
\hline
\end{tabular}
\caption{Struttura della sottorete N2 (contiene N2A)}
\end{table}

\subsubsection{Sottorete N1B (15 hosts)}
Il prossimo blocco disponibile dopo N2 (\texttt{.0} - \texttt{.127}) inizia da \texttt{199.201.17.128}.
\begin{itemize}
    \item Host richiesti: 15. Indirizzi totali: $15 + 2 = 17$.
    \item Potenza di 2 $\geq 17$: $32 = 2^5$. Quindi $h=5$ bit per host.
    \item Prefisso: $32 - 5 = /27$. Netmask: \texttt{255.255.255.224}.
    \item Allocazione N1B: \texttt{199.201.17.128 / 27}.
\end{itemize}

\begin{table}[h]
\centering
\begin{tabular}{|l|c|}
\hline
\rowcolor{bg_custom}
\textcolor{white}{\textbf{Elemento}} & \textcolor{white}{\textbf{Indirizzo}} \\
\hline
\textbf{Network N1B} & \texttt{199.201.17.128} \\
\hline
\textbf{Netmask} & \texttt{255.255.255.224} \\
\hline
\textbf{First Host} & \texttt{199.201.17.129} \\
\hline
\textbf{Last Host} & \texttt{199.201.17.157} \\
\hline
\textbf{Router} & \texttt{199.201.17.158} \\
\hline
\textbf{Broadcast} & \texttt{199.201.17.159} \\
\hline
\end{tabular}
\caption{Struttura della sottorete N1B}
\end{table}

\subsubsection{Sottorete N1A (6 hosts)}
Il prossimo blocco disponibile dopo N1B (\texttt{.128} - \texttt{.159}) inizia da \texttt{199.201.17.160}.
\begin{itemize}
    \item Host richiesti: 6. Indirizzi totali: $6 + 2 = 8$.
    \item Potenza di 2 $\geq 8$: $8 = 2^3$. Quindi $h=3$ bit per host.
    \item Prefisso: $32 - 3 = /29$. Netmask: \texttt{255.255.255.248}.
    \item Allocazione N1A: \texttt{199.201.17.160 / 29}.
\end{itemize}

\begin{table}[h]
\centering
\begin{tabular}{|l|c|}
\hline
\rowcolor{bg_custom}
\textcolor{white}{\textbf{Elemento}} & \textcolor{white}{\textbf{Indirizzo}} \\
\hline
\textbf{Network N1A} & \texttt{199.201.17.160} \\
\hline
\textbf{Netmask} & \texttt{255.255.255.248} \\
\hline
\textbf{First Host} & \texttt{199.201.17.161} \\
\hline
\textbf{Last Host} & \texttt{199.201.17.165} \\
\hline
\textbf{Router} & \texttt{199.201.17.166} \\
\hline
\textbf{Broadcast} & \texttt{199.201.17.167} \\
\hline
\end{tabular}
\caption{Struttura della sottorete N1A}
\end{table}

\subsubsection{Sottorete N1 (64 hosts, include N1A e N1B)}
\begin{itemize}
    \item Host richiesti (totali in N1): 64. Indirizzi totali: $64 + 2 = 66$.
    \item Potenza di 2 $\geq 66$: $128 = 2^7$. Quindi $h=7$ bit per host.
    \item Prefisso: $32 - 7 = /25$. Netmask: \texttt{255.255.255.128}.
    \item N1 ingloba N1B (\texttt{.128} - \texttt{.159}) e N1A (\texttt{.160} - \texttt{.167}). Allocazione N1: \texttt{199.201.17.128 / 25}.
\end{itemize}

\begin{table}[h]
\centering
\begin{tabular}{|l|c|l|}
\hline
\rowcolor{bg_custom}
\textcolor{white}{\textbf{Elemento}} & \textcolor{white}{\textbf{Indirizzo}} & \textcolor{white}{\textbf{Note}} \\
\hline
\textbf{Network N1} & \texttt{199.201.17.128} & Stesso network di N1B ma /25 \\
\hline
\textbf{Netmask} & \texttt{255.255.255.128} & 25 bit a 1, 7 bit a 0 \\
\hline
\textbf{First Host} & \texttt{199.201.17.129} & Coincide con primo host N1B \\
\hline
\textbf{Last Host} & \texttt{199.201.17.252} & Per host in N1 ma non N1A/N1B \\
\hline
\textbf{Router N1} & \texttt{199.201.17.253} & Per host in N1 ma non N1A/N1B \\
\hline
\textbf{Broadcast} & \texttt{199.201.17.255} & Network + $2^7 - 1$ = 128 + 127 \\
\hline
\end{tabular}
\caption{Struttura della sottorete N1 (contiene N1A e N1B)}
\end{table}

\subsection{Visualizzazione della Gerarchia delle Subnet}

\begin{figure}[H]
\centering
\includegraphics[width=0.8\textwidth]{images/subnet_hierarchy.png}
\caption{Diagramma gerarchico delle subnet dell'Esempio 1}
\label{fig:subnet_hierarchy}
\end{figure}

Come mostrato nel diagramma \ref{fig:subnet_hierarchy}, la gerarchia delle subnet riflette la struttura ad albero della suddivisione VLSM. La rete principale N si divide in due grandi blocchi (N1 e N2), ognuno dei quali contiene sottoreti più piccole e specifiche.

\subsection{Mappa Completa degli Indirizzi}

\begin{figure}[H]
\centering
\includegraphics[width=0.8\textwidth]{images/subnet_address_blocks.png}
\caption{Mappa dettagliata di tutti gli indirizzi IP della rete 199.201.17.0/24}
\label{fig:subnet_blocks}
\end{figure}

La figura \ref{fig:subnet_blocks} mostra come ogni indirizzo IP (rappresentato da un quadratino) viene assegnato alle diverse subnet. Questa visualizzazione permette di vedere chiaramente la distribuzione dello spazio di indirizzamento e l'efficienza della suddivisione VLSM.

\subsection{Considerazioni sull'Esempio 1}
\begin{itemize}
    \item \textbf{Conflitto di Broadcast?:} La rete N1 (\texttt{199.201.17.128/25}) ha come broadcast \texttt{199.201.17.255}. Anche la rete padre N (\texttt{199.201.17.0/24}) ha come broadcast \texttt{199.201.17.255}. Questo non è problematico perché ogni host interpreta l'indirizzo di broadcast in base alla propria netmask. Dato che N1 e N2 coprono l'intero spazio /24, non ci sono host "solo in N".
    \item \textbf{Router di Default:}
    \begin{itemize}
        \item Host in N1A usano \texttt{199.201.17.166} come gateway.
        \item Host in N1B usano \texttt{199.201.17.158} come gateway.
        \item Host in N1 (ma non N1A/N1B) usano \texttt{199.201.17.253} come gateway.
        \item I router \texttt{.166} e \texttt{.158} avranno come default gateway \texttt{199.201.17.253}.
        \item Il router di N1 (\texttt{.253}) e il router di N2 (\texttt{.126}) avranno come default gateway il router principale della rete N (\texttt{199.201.17.254} o un router del provider).
    \end{itemize}
\end{itemize}

\newpage
\section{Esempio 2: Rete \texttt{130.136.0.0 / 16}}
Si dispone di un blocco iniziale \texttt{130.136.0.0 / 16}.
\begin{minted}{text}
Netmask: 255.255.0.0
Indirizzi totali: 2^16 = 65536
\end{minted}

\subsection{Requisiti}
\begin{itemize}
    \item LAN IP2: 260 host (divisa in IP2-B: 140 host, IP2-A: 120 host)
    \item LAN IP1: 48 host
    \item LAN IP3: 4 host
\end{itemize}

\subsection{Ordine di allocazione}
Si alloca partendo dalla sottorete con più host richiesti, e si procede con blocchi contigui.

\subsubsection{Sottorete IP2-B (140 hosts)}
\begin{itemize}
    \item Host richiesti: 140. Indirizzi totali: $140 + 2 = 142$.
    \item Potenza di 2 $\geq 142$: $256 = 2^8$. $h=8$ bit per host.
    \item Prefisso: $/24$. Netmask: \texttt{255.255.255.0}.
    \item Allocazione IP2-B: \texttt{130.136.0.0 / 24}.
\end{itemize}
\begin{minted}{text}
Rete IP2-B:
Network:    130.136.0.0
Netmask:    255.255.255.0
First Host: 130.136.0.1
Last Host:  130.136.0.253
Router:     130.136.0.254
Broadcast:  130.136.0.255
\end{minted}

\subsubsection{Sottorete IP2-A (120 hosts)}
Prossimo blocco dopo IP2-B (che finisce a \texttt{130.136.0.255}).
\begin{itemize}
    \item Host richiesti: 120. Indirizzi totali: $120 + 2 = 122$.
    \item Potenza di 2 $\geq 122$: $128 = 2^7$. $h=7$ bit per host.
    \item Prefisso: $/25$. Netmask: \texttt{255.255.255.128}.
    \item Allocazione IP2-A: \texttt{130.136.1.0 / 25}.
\end{itemize}
\begin{minted}{text}
Rete IP2-A:
Network:    130.136.1.0
Netmask:    255.255.255.128
First Host: 130.136.1.1
Last Host:  130.136.1.125
Router:     130.136.1.126
Broadcast:  130.136.1.127
\end{minted}

\subsubsection{Sottorete IP2 (260 hosts, include IP2-A e IP2-B)}
IP2 deve contenere IP2-B (\texttt{130.136.0.0/24}) e IP2-A (\texttt{130.136.1.0/25}).
\begin{itemize}
    \item Range coperto: da \texttt{130.136.0.0} a \texttt{130.136.1.127}. Totale indirizzi usati da IP2-A/B = $256 + 128 = 384$.
    \item Potenza di 2 $\geq 384$: $512 = 2^9$. $h=9$ bit per host.
    \item Prefisso: $/23$. Netmask: \texttt{255.255.254.0}.
    \item Allocazione IP2: \texttt{130.136.0.0 / 23} (copre \texttt{130.136.0.0} - \texttt{130.136.1.255}).
\end{itemize}
\begin{minted}{text}
Rete IP2:
Network:    130.136.0.0
Netmask:    255.255.254.0
First Host: 130.136.0.1 (in IP2-B)
Last Host:  130.136.1.253 (per host in IP2 ma non IP2-A/B)
Router:     130.136.1.254 (per host in IP2 ma non IP2-A/B)
Broadcast:  130.136.1.255
Host in IP2 ma non in IP2-A/B: da 130.136.1.128 a 130.136.1.253
\end{minted}

\subsubsection{Sottorete IP1 (48 hosts)}
Prossimo blocco dopo IP2 (che finisce a \texttt{130.136.1.255}).
\begin{itemize}
    \item Host richiesti: 48. Indirizzi totali: $48 + 2 = 50$.
    \item Potenza di 2 $\geq 50$: $64 = 2^6$. $h=6$ bit per host.
    \item Prefisso: $/26$. Netmask: \texttt{255.255.255.192}.
    \item Allocazione IP1: \texttt{130.136.2.0 / 26}.
\end{itemize}
\begin{minted}{text}
Rete IP1:
Network:    130.136.2.0
Netmask:    255.255.255.192
First Host: 130.136.2.1
Last Host:  130.136.2.61
Router:     130.136.2.62
Broadcast:  130.136.2.63
\end{minted}

\subsubsection{Sottorete IP3 (4 hosts)}
Prossimo blocco dopo IP1 (che finisce a \texttt{130.136.2.63}).
\begin{itemize}
    \item Host richiesti: 4. Indirizzi totali: $4 + 2 = 6$.
    \item Potenza di 2 $\geq 6$: $8 = 2^3$. $h=3$ bit per host.
    \item Prefisso: $/29$. Netmask: \texttt{255.255.255.248}.
    \item Allocazione IP3: \texttt{130.136.2.64 / 29}.
\end{itemize}
\begin{minted}{text}
Rete IP3:
Network:    130.136.2.64
Netmask:    255.255.255.248
First Host: 130.136.2.65
Last Host:  130.136.2.69
Router:     130.136.2.70
Broadcast:  130.136.2.71
\end{minted}

\subsection{Considerazioni sull'Esempio 2 (Default Routers)}
\begin{itemize}
    \item Host in IP2-B (\texttt{130.136.0.0/24}) usano \texttt{130.136.0.254} come gateway.
    \item Host in IP2-A (\texttt{130.136.1.0/25}) usano \texttt{130.136.1.126} come gateway.
    \item Host in IP2 ma fuori da IP2-A/IP2-B (range \texttt{130.136.1.128} - \texttt{130.136.1.253}, con netmask \texttt{/23}) usano \texttt{130.136.1.254} come gateway.
    \item Il router \texttt{130.136.0.254} (di IP2-B) e \texttt{130.136.1.126} (di IP2-A) avranno come default gateway \texttt{130.136.1.254} (router di IP2).
    \item Host in IP1 (\texttt{130.136.2.0/26}) usano \texttt{130.136.2.62} come gateway.
    \item Host in IP3 (\texttt{130.136.2.64/29}) usano \texttt{130.136.2.70} come gateway.
    \item I router di IP1, IP2, IP3 avranno come default gateway il router principale della rete \texttt{130.136.0.0/16} (es. \texttt{130.136.255.254}, se così configurato, o un router del provider).
\end{itemize}

\newpage
\section{Riepilogo Punti Chiave}
\begin{itemize}
    \item Il subnetting e VLSM sono essenziali per un uso efficiente degli indirizzi IP.
    \item Si parte sempre dal numero di host richiesti, si aggiungono 2 (indirizzo di rete e di broadcast), e si trova la potenza di 2 immediatamente superiore o uguale per determinare i bit di host necessari.
    \item L'allocazione dei blocchi di indirizzi dovrebbe essere contigua per evitare la frammentazione dello spazio di indirizzamento e semplificare la gestione.
    \item Ogni sottorete definita ha il suo indirizzo di rete, il suo indirizzo di broadcast e necessita di un router (gateway) per la comunicazione esterna.
    \item La gerarchia delle reti implica che i router delle sottoreti più piccole puntino ai router delle reti che le contengono come loro default gateway, fino al router principale della rete o del provider.
    \item La comprensione della subnet mask è cruciale per ogni host per determinare quali altri host sono nella sua stessa sottorete e quale indirizzo usare come broadcast.
\end{itemize}

\end{document}